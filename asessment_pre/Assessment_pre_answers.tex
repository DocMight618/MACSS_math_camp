\documentclass[11pt]{article}

\usepackage{amsmath}
\usepackage{amsfonts}
\usepackage{amssymb}
\usepackage{amsthm}
\usepackage{graphicx}
\usepackage{setspace}
\usepackage{fullpage}
\usepackage{color}
\usepackage{xcolor,colortbl}
\usepackage{comment}
\usepackage{bm}
\usepackage{url}
\usepackage{hyperref}
\usepackage{color,hyperref}
\definecolor{darkblue}{rgb}{0.0,0.0,0.3}
\hypersetup{colorlinks,breaklinks,
            linkcolor=darkblue,urlcolor=darkblue,
            anchorcolor=darkblue,citecolor=darkblue}
            

%Change to Helvetica
%\usepackage[scaled]{helvet}
%\renewcommand*\familydefault{\sfdefault} %% Only if the base font of the document is to be sans serif
%\usepackage[T1]{fontenc}

%Change to fourier
%\usepackage{fourier}
%\usepackage[T1]{fontenc} 

\usepackage[T1]{fontenc}
\usepackage{charter}
\usepackage[expert]{mathdesign}


\begin{document}
\noindent\textbf{Assessment Evaluation  - POST Math Prefresher - 9/2017  ANSWER KEY} \\
We've just spent a magical week together. This is to see how much stuck from the course and to help you identify areas that could use more attention and development. Do your best.  \textbf{Note: THERE ARE TWO SIDES!} \\

\noindent \textbf{Background questions:} \\
\indent Name, discipline and subfield? \\

\vspace{3mm}
\noindent \textbf{Questions}

\begin{enumerate}

\item Basics
\begin{enumerate}
\item Explain the significance or use of the following symbols:
\begin{enumerate} 
\item $\Pi$ \color{gray} ANS: This is for multiplication of a series or the symbol to represent 3.14... \color{black}
\item $\Sigma$  \color{gray} ANS: This is for the summation of a series  \color{black}
\end{enumerate}


\item Solve: (no need to simplify but show steps/work if possible)
\begin{enumerate}
\item $4\geq x-7$  \color{gray} ANS: $x \leq 11$ \color{black} \\
\item $-9x+2>3$ \color{gray} ANS: $x < \frac{-1}{9}$ \color{black} \\
\item $|x-2|\leq2$  \color{gray} ANS: $0 \leq x \leq 4$ \color{black} \\
\item $2e^{6x}=18$  \color{gray} ANS: $e^{6x}=9$, $x=\frac{ln(9)}{6}$ \color{black} \\
\item $e^{x^2}=1$ \color{gray} ANS: $x^2=0$ so $x=0$ \color{black} \\
\item $ln(x^2)=5$ \color{gray} ANS: $ln(x)=2.5$, so $x=e^2.5$\color{black} \\
\item $\sum_{n=1}^{10} 3+n $ \color{gray} ANS: $3*10+(10*(11))/2=30+55=85$ \color{black} \\
\item $4!$\color{gray} ANS: $4*3*2*1$ \color{black} \\
\item $(\frac{x^4y^{-3}}{x^2y^3})^3$ \color{gray} ANS: $(\frac{x^2}{y^6})3=\frac{x^6}{y^18}$\color{black} \\
\end{enumerate}

\item Factor
\begin{enumerate}
\item $m^2+3m+2 $ \color{gray} ANS: $(m+2)(m+1)$\color{black} \\
\item $x^2+7x+6$ \color{gray} ANS: $(x+1)(x+6)$ \color{black} \\
\item $x^4+x^2$ \color{gray} ANS: $x^2(x^2+1) $ \color{black} \\
\end{enumerate}

\end{enumerate}

\item Set Theory
\begin{enumerate}
\item Explain the meaning of the following symbols:
\begin{enumerate}
\item $\in$ \color{gray} ANS: `an element of' \color{black} \\
\item $\forall$ \color{gray} ANS: `for all'  \color{black} \\
\end{enumerate}
\item Suppose $A=\{3, 6, 12\}$,  $B=\{\text{hat, bulldozer, forklift}\}$ and $C=\{x| x\text{ is a natural number}| x >3 \text{ and } x<9\}$
\begin{enumerate}
\item What is $A \cup B$? \color{gray} ANS: $\{ 3, 6, 12,  \text{hat, bulldozer, forklift}\}$\color{black} \\
\item Write the elements of C \color{gray} ANS: $\{ 4,5,6,7,8\} $ \color{black} \\
\item What is $A \cap C$? \color{gray} ANS: $\{6\}$ \color{black} \\
\item What is $A\setminus C$? \color{gray} ANS: $\{3,12\}$ \color{black} \\
\end{enumerate}
\end{enumerate}



\item Functions \& Pre-Calculus
\begin{enumerate}
\item What is a continuous function?  \color{gray} ANS: One you can draw without picking up a pencil -- where the limit from the left equals that from the right and equals the value at the point (and the value exists!) \color{black} \\
\item Draw an increasing function. \color{gray} ANS: one where y gets larger as x increases (x and y move together--positive slope) \color{black} \\
\newline
\item What is a tangent line? What does it do? \color{gray} ANS: The tangent line is one that touches the graph of a function at only one point.  \color{black} \\
\item What is a derivative?  \color{gray} ANS: an instantaneous rate of change -- the slope of the tangent line at a particular point \color{black} \\
\end{enumerate}

\item Matrix Algebra
\begin{enumerate}
\item Give an example of a $3 \times 4$ matrix  \color{gray}
$\begin{bmatrix}
  a & d & g & j \\
  b & e & h & k \\
  c & f & i & l \\
\end{bmatrix}$ \\ \color{black}
\item Consider the following matrices: \\
\textbf{A} =
$
\begin{bmatrix}
  3 & 4 & 1 \\
  0 & 2 & 1 \\
\end{bmatrix}$
\textbf{B} =
$\begin{bmatrix}
  0 & 1 & 1 \\
  1 & 2 & 4 \\
\end{bmatrix}$
\textbf{C} =
$\begin{bmatrix}
  1 & 4 & 1 \\
  0 & 2 & 1 \\
  6 & 2 & 9 \\
\end{bmatrix}$ \\
\begin{enumerate}
\item Which matrices can be added together?  \color{gray} $A$ and $B$ can be added together because they have the same dimensions $2 \times 3$ \color{black}
\item Add the matrices from the above response.  \color{gray} 
$\begin{bmatrix}
  3 & 5 & 2 \\
  1 & 4 & 5 \\
\end{bmatrix}$

\color{black} 
\item Which matrices can be multiplied together?  \color{gray} ANS: $A$ and $B$ can each be multiplied by $C$ (SO, $A*C$) but cannot multiply $A$ and $B$ by each other or $C$ by $A$ or $B$ (so, yes AC, no CA) We know that because $A$ is 2x3 and $C$ is 3x3, we can multiply them (middle two numbers same). We also know that the dimensions of the final matrix will be the number of rows in $A$ (2) and the number of columns in $C$ (3) -- so 2x3.  \color{black}
\item Multiply the matrices from the above response. \color{gray}
Can do $A*C$ or $B*C$. $A*C$ as example:
$\begin{bmatrix}
 (3*1+0+1*6) & (3*4+4*2+1*2) & (3*1+4*1+1*9) \\
 (0+0+6*1) & (0+2*2+1*2) & (0+2*1+1*9)\\
\end{bmatrix}$

\color{black}
\end{enumerate}


\end{enumerate}

\item Calculus 
\begin{enumerate}
\item what is the derivative of $4$? \color{gray} ANS: $0$ \color{black} \\
\item what is the derivative of $2x$?  \color{gray} ANS: $2$ \color{black} \\
\item calculate the derivative of $7m^2-m+2$  \color{gray} ANS: $14m-1$ \color{black}\\
\item calculate the integral $\int_{0}^{5}  (x^3+0.5x^2+5x)dx$  \color{gray} ANS: $x^4/4+x^3/6+5/2x^2|_0^5$\color{black}\\
\item calculate the integral $\int e^x dx$  \color{gray} ANS: $e^x+c$ \color{black} \\
\end{enumerate}



\end{enumerate} 



\end{document}