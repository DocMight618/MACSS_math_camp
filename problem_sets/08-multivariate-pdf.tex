% Options for packages loaded elsewhere
\PassOptionsToPackage{unicode}{hyperref}
\PassOptionsToPackage{hyphens}{url}
\documentclass[
]{article}
\usepackage{xcolor}
\usepackage[margin=1in]{geometry}
\usepackage{amsmath,amssymb}
\setcounter{secnumdepth}{5}
\usepackage{iftex}
\ifPDFTeX
  \usepackage[T1]{fontenc}
  \usepackage[utf8]{inputenc}
  \usepackage{textcomp} % provide euro and other symbols
\else % if luatex or xetex
  \usepackage{unicode-math} % this also loads fontspec
  \defaultfontfeatures{Scale=MatchLowercase}
  \defaultfontfeatures[\rmfamily]{Ligatures=TeX,Scale=1}
\fi
\usepackage{lmodern}
\ifPDFTeX\else
  % xetex/luatex font selection
\fi
% Use upquote if available, for straight quotes in verbatim environments
\IfFileExists{upquote.sty}{\usepackage{upquote}}{}
\IfFileExists{microtype.sty}{% use microtype if available
  \usepackage[]{microtype}
  \UseMicrotypeSet[protrusion]{basicmath} % disable protrusion for tt fonts
}{}
\makeatletter
\@ifundefined{KOMAClassName}{% if non-KOMA class
  \IfFileExists{parskip.sty}{%
    \usepackage{parskip}
  }{% else
    \setlength{\parindent}{0pt}
    \setlength{\parskip}{6pt plus 2pt minus 1pt}}
}{% if KOMA class
  \KOMAoptions{parskip=half}}
\makeatother
\usepackage{graphicx}
\makeatletter
\newsavebox\pandoc@box
\newcommand*\pandocbounded[1]{% scales image to fit in text height/width
  \sbox\pandoc@box{#1}%
  \Gscale@div\@tempa{\textheight}{\dimexpr\ht\pandoc@box+\dp\pandoc@box\relax}%
  \Gscale@div\@tempb{\linewidth}{\wd\pandoc@box}%
  \ifdim\@tempb\p@<\@tempa\p@\let\@tempa\@tempb\fi% select the smaller of both
  \ifdim\@tempa\p@<\p@\scalebox{\@tempa}{\usebox\pandoc@box}%
  \else\usebox{\pandoc@box}%
  \fi%
}
% Set default figure placement to htbp
\def\fps@figure{htbp}
\makeatother
\setlength{\emergencystretch}{3em} % prevent overfull lines
\providecommand{\tightlist}{%
  \setlength{\itemsep}{0pt}\setlength{\parskip}{0pt}}
\usepackage{bookmark}
\IfFileExists{xurl.sty}{\usepackage{xurl}}{} % add URL line breaks if available
\urlstyle{same}
\hypersetup{
  pdftitle={PSET 8: Multivariate distributions},
  hidelinks,
  pdfcreator={LaTeX via pandoc}}

\title{PSET 8: Multivariate distributions}
\author{}
\date{\vspace{-2.5em}}

\begin{document}
\maketitle

\textbf{Note: all homework uploads should be as a PDF or image
\emph{and} have the questions identified}. We'll be giving zero credit
for submissions that don't follow this protocol as it adds considerable
time to grading. Thank you!

\begin{itemize}
\item
  Name
\item
  How long did this problem set take you?
\item
  How difficult was this problem set? very easy 1 2 3 4 5 very
  challenging
\end{itemize}

\section{Calculating conditional PDF}\label{calculating-conditional-pdf}

Let \(f(x,y) = 5x^2y^2\) for \(0 \leq x \leq y \leq 1\). Find
\(f(x|y)\).\footnote{Inspired by Grimmer HW12.4}

\hfill\break
\hfill\break
\hfill\break
\hfill\break
\hfill\break
\hfill\break
\hfill\break
\hfill\break
\hfill\break

\section{Properties of a joint PDF}\label{properties-of-a-joint-pdf}

Continuous random variables \(X\) and \(Y\) have the following joint
probability density function (PDF):\footnote{Inspired by Grimmer HW12.1}

\[
\begin{aligned}
f_{XY} (x, y) &= \begin{cases}
k x^3 y^2 & \text{where} \; 0 < x, y < 6 \\
0 & \text{otherwise}
\end{cases}
\end{aligned}
\]

\begin{quote}
Note: \(0 < x,y <6\) means that both \(x\) and \(y\) are between 0 and
6; it does not mean that \(x\) is greater than 0 and \(y\) is less than
6.
\end{quote}

\begin{enumerate}
\def\labelenumi{\alph{enumi}.}
\item
  Find \(k\).

  \strut \\
  \strut \\
  \strut \\
  \strut \\
  \strut \\
  \strut \\
\item
  Find the marginal PDF of \(X\), \(f_X (x)\).

  \hfill\break
  \hfill\break
  \hfill\break
  \hfill\break
  \hfill\break
  \hfill\break
  \hfill\break
  \hfill\break
  \hfill\break
  \hfill\break
\item
  Find the marginal PDF of \(Y\), \(f_Y (y)\).

  \strut \\
  \strut \\
  \strut \\
  \strut \\
  \strut \\
  \strut \\
  \strut \\
  \strut \\
  \strut \\
  \strut \\
\item
  Find E\([X]\).

  \strut \\
  \strut \\
  \strut \\
  \strut \\
  \strut \\
  \strut \\
  \strut \\
  \strut \\
\item
  Find E\([Y]\).

  \strut \\
  \strut \\
  \strut \\
  \strut \\
  \strut \\
  \strut \\
  \strut \\
  \strut \\
\item
  Find \(Var(X)\).

  \strut \\
  \strut \\
  \strut \\
  \strut \\
  \strut \\
  \strut \\
  \strut \\
  \strut \\
  \strut \\
  \strut \\
  \strut \\
  \strut \\
  \strut \\
  \strut \\
\item
  Find \(Var(Y)\).

  \strut \\
  \strut \\
  \strut \\
  \strut \\
  \strut \\
  \strut \\
  \strut \\
  \strut \\
\item
  Find \(Cov(X, Y)\).

  \hfill\break
  \hfill\break
  \hfill\break
  \hfill\break
  \hfill\break
  \hfill\break
  \hfill\break
  \hfill\break
  \hfill\break
  \hfill\break
\item
  Are \(X\) and \(Y\) independent? Explain your reasoning using
  mathematical concepts from the course.

  \hfill\break
  \hfill\break
  \hfill\break
  \hfill\break
  \hfill\break
  \hfill\break
  \hfill\break
  \hfill\break
  \hfill\break
  \hfill\break
\item
  What is the PDF of \(X\) conditional on \(Y, f_{X|Y} (x|y)\)?

  \hfill\break
  \hfill\break
  \hfill\break
  \hfill\break
  \hfill\break
  \hfill\break
  \hfill\break
  \hfill\break
  \hfill\break
  \hfill\break
  \hfill\break
  \hfill\break
  \hfill\break
\item
  What is the PDF of \(Y\) conditional on \(X, f_{Y|X} (y|x)\)?

  \hfill\break
  \hfill\break
  \hfill\break
  \hfill\break
  \hfill\break
  \hfill\break
  \hfill\break
  \hfill\break
\end{enumerate}

\section[Properties of joint random
variables]{\texorpdfstring{Properties of joint random
variables\footnote{Inspired by Grimmer HW12.3}}{Properties of joint random variables}}\label{properties-of-joint-random-variables}

Suppose the following:

\begin{itemize}
\tightlist
\item
  \(E[D] = 8\)
\item
  \(E[F] = 4\)
\item
  \(E[DF] = 10\)
\item
  \(Var(D) = 30\)
\item
  \(Var(F) = 60\)
\end{itemize}

\begin{enumerate}
\def\labelenumi{\alph{enumi}.}
\item
  What is \(Cov(D,F)\)?

  \hfill\break
  \hfill\break
  \hfill\break
  \hfill\break
  \hfill\break
  \hfill\break
  \hfill\break
\item
  What is the correlation between \(D\) and \(F\)?

  \strut \\
  \strut \\
  \strut \\
  \strut \\
  \strut \\
  \strut \\
  \strut \\
\item
  Suppose you multiplied \(F\) by 2 to generate a new variable, \(H\).
  What is \(Cov(D,H)\)?

  \hfill\break
  \hfill\break
  \hfill\break
  \hfill\break
  \hfill\break
  \hfill\break
  \hfill\break
  \hfill\break
  \hfill\break
  \hfill\break
\item
  What is \(Cor(D,H)\)? How does this compare to your answer to Part (b)
  of this question?

  \hfill\break
  \hfill\break
  \hfill\break
  \hfill\break
  \hfill\break
  \hfill\break
  \hfill\break
  \hfill\break
  \hfill\break
  \hfill\break
\item
  Suppose instead that \(Var(D) = 40\). How would this change
  \(Cor(D,F)\)?

  \hfill\break
  \hfill\break
  \hfill\break
  \hfill\break
  \hfill\break
  \hfill\break
  \hfill\break
  \hfill\break
  \hfill\break
  \hfill\break
\end{enumerate}

\section{Continuous Bayes' theorem}\label{continuous-bayes-theorem}

Previously, we used Bayes' theorem to link the conditional probability
of discrete events \(A\) given \(B\) to the probability of \(B\) given
\(A\). There is an analogous Bayes' theorem that relates the conditional
densities of random variables \(X\) and \(\theta\) (below) Prove the
continuous Bayes' theorem.\footnote{Inspired by Grimmer HW12.5}

\[f(\theta \mid X) = \frac{f(X \mid \theta) f(\theta)}{\int f(X \mid \theta) f(\theta)d\theta}\]

\hfill\break
\hfill\break
\hfill\break
\hfill\break
\hfill\break
\hfill\break
\hfill\break
\hfill\break

\section{AI and Resources statement}\label{ai-and-resources-statement}

\begin{itemize}
\tightlist
\item
  Please list (in detail) all resources you used for this assignment. If
  you worked with people, list them here as well. It is not enough to
  say that you used a resource for help, you need to be specific on the
  link and \emph{how} it was helpful. W/R/T gen AI tools (including GPT,
  etc. ) you cannot use them to do work on your behalf -- you cannot put
  in any of the questions, etc. You can ask for help on logic / sample
  problems. If you do use GPT or other AI tools, you need to provide a
  link to your chat transcript. Any suspected academic integrity
  violations will be immediately reported.
\end{itemize}

\subsection{(Optional -- complete elsewhere) Submission of practice
questions}\label{optional-complete-elsewhere-submission-of-practice-questions}

Submit practice questions for the final exam here:
\url{https://forms.gle/CPo9FMQgQRPePDfN7} Note that we need at least 10
people to submit before there's enough to circulate!

\end{document}
