% Options for packages loaded elsewhere
\PassOptionsToPackage{unicode}{hyperref}
\PassOptionsToPackage{hyphens}{url}
\documentclass[
]{article}
\usepackage{xcolor}
\usepackage[margin=1in]{geometry}
\usepackage{amsmath,amssymb}
\setcounter{secnumdepth}{5}
\usepackage{iftex}
\ifPDFTeX
  \usepackage[T1]{fontenc}
  \usepackage[utf8]{inputenc}
  \usepackage{textcomp} % provide euro and other symbols
\else % if luatex or xetex
  \usepackage{unicode-math} % this also loads fontspec
  \defaultfontfeatures{Scale=MatchLowercase}
  \defaultfontfeatures[\rmfamily]{Ligatures=TeX,Scale=1}
\fi
\usepackage{lmodern}
\ifPDFTeX\else
  % xetex/luatex font selection
\fi
% Use upquote if available, for straight quotes in verbatim environments
\IfFileExists{upquote.sty}{\usepackage{upquote}}{}
\IfFileExists{microtype.sty}{% use microtype if available
  \usepackage[]{microtype}
  \UseMicrotypeSet[protrusion]{basicmath} % disable protrusion for tt fonts
}{}
\makeatletter
\@ifundefined{KOMAClassName}{% if non-KOMA class
  \IfFileExists{parskip.sty}{%
    \usepackage{parskip}
  }{% else
    \setlength{\parindent}{0pt}
    \setlength{\parskip}{6pt plus 2pt minus 1pt}}
}{% if KOMA class
  \KOMAoptions{parskip=half}}
\makeatother
\usepackage{longtable,booktabs,array}
\usepackage{calc} % for calculating minipage widths
% Correct order of tables after \paragraph or \subparagraph
\usepackage{etoolbox}
\makeatletter
\patchcmd\longtable{\par}{\if@noskipsec\mbox{}\fi\par}{}{}
\makeatother
% Allow footnotes in longtable head/foot
\IfFileExists{footnotehyper.sty}{\usepackage{footnotehyper}}{\usepackage{footnote}}
\makesavenoteenv{longtable}
\usepackage{graphicx}
\makeatletter
\newsavebox\pandoc@box
\newcommand*\pandocbounded[1]{% scales image to fit in text height/width
  \sbox\pandoc@box{#1}%
  \Gscale@div\@tempa{\textheight}{\dimexpr\ht\pandoc@box+\dp\pandoc@box\relax}%
  \Gscale@div\@tempb{\linewidth}{\wd\pandoc@box}%
  \ifdim\@tempb\p@<\@tempa\p@\let\@tempa\@tempb\fi% select the smaller of both
  \ifdim\@tempa\p@<\p@\scalebox{\@tempa}{\usebox\pandoc@box}%
  \else\usebox{\pandoc@box}%
  \fi%
}
% Set default figure placement to htbp
\def\fps@figure{htbp}
\makeatother
\setlength{\emergencystretch}{3em} % prevent overfull lines
\providecommand{\tightlist}{%
  \setlength{\itemsep}{0pt}\setlength{\parskip}{0pt}}
\usepackage{bookmark}
\IfFileExists{xurl.sty}{\usepackage{xurl}}{} % add URL line breaks if available
\urlstyle{same}
\hypersetup{
  pdftitle={PSET 9: Classical statistical inference},
  hidelinks,
  pdfcreator={LaTeX via pandoc}}

\title{PSET 9: Classical statistical inference}
\author{}
\date{\vspace{-2.5em}}

\begin{document}
\maketitle

\textbf{Note: all homework uploads should have the questions
identified}. We'll be giving zero credit for submissions that don't
follow this protocol as it adds considerable time to grading. Thank you!

\begin{itemize}
\item
  Name
\item
  How long did this problem set take you?
\item
  How difficult was this problem set? very easy 1 2 3 4 5 very
  challenging
\end{itemize}

\section{Properties of estimators}\label{properties-of-estimators}

\begin{enumerate}
\def\labelenumi{\alph{enumi}.}
\item
  Let \(X_1, \ldots, X_n \sim \text{Poisson}(\lambda)\) and let
  \(\hat{\lambda} = \dfrac{\sum_{i=1}^n X_i}{2n}\). Find the bias,
  standard error, and MSE of this estimator.\footnote{Inspired by
    Wasserman 6.6.1}
\item
  Let \(X_1, \ldots, X_n \sim \text{Uniform}(0, \theta)\) and let
  \(\hat{\theta} = 3 \bar{X}_n\). Find the bias, standard error, and MSE
  of this estimator.\footnote{Inspired by Wasserman 6.6.3}
\end{enumerate}

\section{Social newsing}\label{social-newsing}

A poll conducted in 2025 found that 12\% of U.S. adults users get at
least some news on TikTok. The standard error for this estimate was
2.4\%, and a normal distribution may be used to model the sample
proportion.\footnote{Inspired byOI 4.8 and 4.10}

\begin{enumerate}
\def\labelenumi{\alph{enumi}.}
\tightlist
\item
  Construct a 95\% confidence interval for the fraction of U.S. adult
  users who get some news on TikTok, and interpret the confidence
  interval in context.
\end{enumerate}

\section{(T/F) Social Newsing}\label{tf-social-newsing}

Identify the following statements as true or false using the above
information. Provide an explanation to justify each of your answers.
Explain in approx 75 words.

\begin{enumerate}
\def\labelenumi{\alph{enumi}.}
\item
  The data provide statistically significant evidence that more than
  16\% of U.S. adults get some news through TikTok Use a significance
  level of \(\alpha = 0.05\).
\item
  Since the standard error is 2.4\%, we can conclude that 97.6\% of all
  U.S. adults were included in the study.
\item
  If we want to reduce the standard error of the estimate, we should
  collect less data.
\item
  If we construct a 90\% confidence interval for the percentage of U.S.
  adults who get some news through TikTok, this confidence interval will
  be wider than a corresponding 99\% confidence interval.
\end{enumerate}

\section{Dating on college campuses}\label{dating-on-college-campuses}

A survey conducted on a reasonably random sample of 200 undergraduates
asked, among many other questions, about the number of exclusive
relationships these students have been in. The histogram below shows the
distribution of the data from this sample.

The sample average is 3.2 with a standard deviation of 2.17.

\begin{center}\includegraphics[width=0.9\linewidth]{09-frequentist-inference_files/figure-latex/survey-1} \end{center}

Estimate the average number of exclusive relationships undergraduate
students have been in using the Normal distribution and a 95\%
confidence interval and interpret this interval in context.\footnote{Inspired
  byOI 4.15}

\section{Statistical significance}\label{statistical-significance}

Determine whether the following statement is true or false, and explain
your reasoning: ``With large sample sizes, even small differences
between the null value and the point estimate can be statistically
significant.''\footnote{Inspired byOI 4.47}

\section{Sleep deprivation}\label{sleep-deprivation}

New York is known as ``the city that never sleeps''. A random sample of
25 New Yorkers were asked how much sleep they get per night. Statistical
summaries of these data are shown below. Do these data provide strong
evidence that New Yorkers sleep less than 8 hours a night on
average?\footnote{Inspired byOI 5.7}

\begin{longtable}[]{@{}lllll@{}}
\toprule\noalign{}
\(n\) & \(\bar{x}\) & \(s\) & \(\min\) & \(\max\) \\
\midrule\noalign{}
\endhead
\bottomrule\noalign{}
\endlastfoot
\(25\) & \(7.63\) & \(0.87\) & \(6.17\) & \(9.78\) \\
\end{longtable}

\begin{enumerate}
\def\labelenumi{\alph{enumi}.}
\item
  Write the hypotheses in symbols and in words.
\item
  Calculate the test statistic, \(T\), and the associated degrees of
  freedom.
\item
  Find and interpret the p-value in this context.
\item
  What is the conclusion of the hypothesis test?
\item
  If you were to construct a 90\% confidence interval that corresponded
  to this hypothesis test, would you expect 8 hours to be in the
  interval?
\end{enumerate}

\section{Interpreting public opinion
polls}\label{interpreting-public-opinion-polls}

On June 28, 2012 the U.S. Supreme Court upheld the much debated 2010
healthcare law, declaring it constitutional. A Gallup poll released the
day after this decision indicates that 46\% of 1,012 Americans agree
with this decision. At a 95\% confidence level, this sample has a 3\%
margin of error. Based on this information, determine if the following
statements are true or false, and explain your reasoning.\footnote{Inspired
  by OI 6.6}

\begin{enumerate}
\def\labelenumi{\alph{enumi}.}
\item
  We are 95\% confident that between 43\% and 49\% of Americans in this
  sample support the decision of the U.S. Supreme Court on the 2010
  healthcare law.
\item
  We are 95\% confident that between 43\% and 49\% of Americans support
  the decision of the U.S. Supreme Court on the 2010 healthcare law.
\item
  If we considered many random samples of 1,012 Americans, and we
  calculated the sample proportions of those who support the decision of
  the U.S. Supreme Court, 95\% of those sample proportions will be
  between 43\% and 49\%.
\item
  The margin of error at a 90\% confidence level would be higher than
  3\%.
\end{enumerate}

\section{Power and sample size.}\label{power-and-sample-size.}

Suppose you have two samples: one with a sample mean of 12, standard
deviation of of 2.13 and \(n_1\) of 87 and a second with a sample mean
of 12.8, standard deviation of 3.25 and an \(n_2\) of 89.

\begin{enumerate}
\def\labelenumi{\alph{enumi}.}
\tightlist
\item
  Is it statistically significant?
\item
  What is the smallest \(n_2\) you could have if you want to reject the
  null hypothesis at \(\alpha < 0.05\)?
\item
  Darn it! Your \(n_2\) is actually only 85! What should your goal be:
  smaller sample deviation by 10\% or larger sample by 10? Explain your
  reasoning and provide mathematical support.
\end{enumerate}

\section{AI and Resources statement}\label{ai-and-resources-statement}

\begin{itemize}
\tightlist
\item
  Please list (in detail) all resources you used for this assignment. If
  you worked with people, list them here as well. It is not enough to
  say that you used a resource for help, you need to be specific on the
  link and \emph{how} it was helpful. W/R/T gen AI tools (including GPT,
  etc. ) you cannot use them to do work on your behalf -- you cannot put
  in any of the questions, etc. You can ask for help on logic / sample
  problems. If you do use GPT or other AI tools, you need to provide a
  link to your chat transcript. Any suspected academic integrity
  violations will be immediately reported.
\end{itemize}

\subsection{Survey Qs:}\label{survey-qs}

\begin{enumerate}
\def\labelenumi{\arabic{enumi}.}
\item
  How prepared do you feel for the final exam? not prepared (1) 2 3 4
  (5) very prepared
\item
  Which content area of the course was the easiest for you? Why? Please
  be specific.
\item
  Which content area of the course was the most challenging for you?
  Why? Please be specific.
\end{enumerate}

\end{document}
