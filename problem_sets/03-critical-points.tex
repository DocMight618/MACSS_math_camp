% Options for packages loaded elsewhere
\PassOptionsToPackage{unicode}{hyperref}
\PassOptionsToPackage{hyphens}{url}
\documentclass[
]{article}
\usepackage{xcolor}
\usepackage[margin=1in]{geometry}
\usepackage{amsmath,amssymb}
\setcounter{secnumdepth}{5}
\usepackage{iftex}
\ifPDFTeX
  \usepackage[T1]{fontenc}
  \usepackage[utf8]{inputenc}
  \usepackage{textcomp} % provide euro and other symbols
\else % if luatex or xetex
  \usepackage{unicode-math} % this also loads fontspec
  \defaultfontfeatures{Scale=MatchLowercase}
  \defaultfontfeatures[\rmfamily]{Ligatures=TeX,Scale=1}
\fi
\usepackage{lmodern}
\ifPDFTeX\else
  % xetex/luatex font selection
\fi
% Use upquote if available, for straight quotes in verbatim environments
\IfFileExists{upquote.sty}{\usepackage{upquote}}{}
\IfFileExists{microtype.sty}{% use microtype if available
  \usepackage[]{microtype}
  \UseMicrotypeSet[protrusion]{basicmath} % disable protrusion for tt fonts
}{}
\makeatletter
\@ifundefined{KOMAClassName}{% if non-KOMA class
  \IfFileExists{parskip.sty}{%
    \usepackage{parskip}
  }{% else
    \setlength{\parindent}{0pt}
    \setlength{\parskip}{6pt plus 2pt minus 1pt}}
}{% if KOMA class
  \KOMAoptions{parskip=half}}
\makeatother
\usepackage{graphicx}
\makeatletter
\newsavebox\pandoc@box
\newcommand*\pandocbounded[1]{% scales image to fit in text height/width
  \sbox\pandoc@box{#1}%
  \Gscale@div\@tempa{\textheight}{\dimexpr\ht\pandoc@box+\dp\pandoc@box\relax}%
  \Gscale@div\@tempb{\linewidth}{\wd\pandoc@box}%
  \ifdim\@tempb\p@<\@tempa\p@\let\@tempa\@tempb\fi% select the smaller of both
  \ifdim\@tempa\p@<\p@\scalebox{\@tempa}{\usebox\pandoc@box}%
  \else\usebox{\pandoc@box}%
  \fi%
}
% Set default figure placement to htbp
\def\fps@figure{htbp}
\makeatother
\setlength{\emergencystretch}{3em} % prevent overfull lines
\providecommand{\tightlist}{%
  \setlength{\itemsep}{0pt}\setlength{\parskip}{0pt}}
\usepackage{bookmark}
\IfFileExists{xurl.sty}{\usepackage{xurl}}{} % add URL line breaks if available
\urlstyle{same}
\hypersetup{
  pdftitle={PSET 3: Critical points and approximation},
  hidelinks,
  pdfcreator={LaTeX via pandoc}}

\title{PSET 3: Critical points and approximation}
\author{}
\date{\vspace{-2.5em}}

\begin{document}
\maketitle

\section{Assignment Qs}\label{assignment-qs}

\begin{itemize}
\item
  Name
\item
  How long did this problem set take you?
\item
  How difficult was this problem set? very easy 1 2 3 4 5 very
  challenging
\end{itemize}

\section{Sketch a function}\label{sketch-a-function}

Sketch the graph of a function (any function you like, no need to
specify a functional form) that is:\footnote{inspired by Grimmer HW3.1}

\begin{enumerate}
\def\labelenumi{\alph{enumi}.}
\tightlist
\item
  Continuous on \([0,3]\) and has the following properties: an absolute
  maximum at 0, an absolute minimum at 3, a local maximum at 1 and a
  local minimum at 2.
\item
  Do the same for another function with the following properties: 4 is a
  \textbf{critical number} (i.e.~\(f'(x) = 0\) or \(f'(x)\) is
  undefined), but there is no local minimum and no local maximum.
\end{enumerate}

\section{Find critical values}\label{find-critical-values}

Find the critical values of these functions:\footnote{inspired by
  Grimmer HW3.2}

\begin{enumerate}
\def\labelenumi{\alph{enumi}.}
\item
  \(f(x) = 5x^{2/3} - 4x\)
\item
  \(s(t) = 3t^4 - 4t^3 + 6t^2\)
\item
  \(f(r) = \dfrac{r}{r^2 +r + 1}\)
\item
  \(h(x) = x \ln(x)\)
\end{enumerate}

\section{Find absolute minimum/maximum
values}\label{find-absolute-minimummaximum-values}

Find the absolute minimum and absolute maximum values of the functions
on the given interval:\footnote{inspired by Grimmer HW3.3}

\begin{enumerate}
\def\labelenumi{\alph{enumi}.}
\item
  \(f(x) = 3x^2 - 12x + 5, [0,1]\)
\item
  \(f(t) = t^2\sqrt{9 - t^2}, [-1,4]\)
\item
  \(s(x) = x - \ln(x), [1/2, 2]\)
\end{enumerate}

\section{Approximate root-finding}\label{approximate-root-finding}

Show that the equation

\[x^7 + 6x - 4 = 0\]

has a root between \(0\) and \(1\).\footnote{inspired by Pemberton and
  Rau 10.1.3}

\begin{enumerate}
\def\labelenumi{\alph{enumi}.}
\item
  Find an initial approximation by ignoring the term \(x^7\).
\item
  Use Newton's method to find the root correct to 3 decimal places.
\end{enumerate}

\section{Apply the mean value
theorem}\label{apply-the-mean-value-theorem}

Does a continuous, differentiable function exist on \([0,4]\) such that
\(f(0) = -1\), \(f(4) = 4\), and \(f'(x) \le 2 \  \forall \, x\)? Use
the mean value theorem to explain your answer.\footnote{inspired by
  Grimmer HW3.5}

\subsection{Optional!: Finding Max/Min}\label{optional-finding-maxmin}

\begin{enumerate}
\def\labelenumi{\alph{enumi}.}
\item
  \textbf{OPTIONAL} \(h(p) = 1 - e^{-p}, [0,1000]\)
\item
  \textbf{OPTIONAL} Demonstrate that the function
  \(f(x) = x^5 + x^3 + x + 1\) has no local maximum and no local
  minimum.\footnote{inspired by Grimmer HW3.4}
\end{enumerate}

\subsection{AI and Resources
statement}\label{ai-and-resources-statement}

\begin{itemize}
\tightlist
\item
  Please list (in detail) all resources you used for this assignment. If
  you worked with people, list them here as well. It is not enough to
  say that you used a resource for help, you need to be specific on the
  link and \emph{how} it was helpful. W/R/T gen AI tools (including GPT,
  etc. ) you cannot use them to do work on your behalf -- you cannot put
  in any of the questions, etc. You can ask for help on logic / sample
  problems. If you do use GPT or other AI tools, you need to provide a
  link to your chat transcript. Any suspected academic integrity
  violations will be immediately reported.
\end{itemize}

\end{document}
