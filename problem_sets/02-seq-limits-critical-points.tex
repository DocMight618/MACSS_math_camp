% Options for packages loaded elsewhere
\PassOptionsToPackage{unicode}{hyperref}
\PassOptionsToPackage{hyphens}{url}
\PassOptionsToPackage{dvipsnames,svgnames,x11names}{xcolor}
%
\documentclass[
  letterpaper,
  DIV=11,
  numbers=noendperiod]{scrartcl}

\usepackage{amsmath,amssymb}
\usepackage{lmodern}
\usepackage{iftex}
\ifPDFTeX
  \usepackage[T1]{fontenc}
  \usepackage[utf8]{inputenc}
  \usepackage{textcomp} % provide euro and other symbols
\else % if luatex or xetex
  \usepackage{unicode-math}
  \defaultfontfeatures{Scale=MatchLowercase}
  \defaultfontfeatures[\rmfamily]{Ligatures=TeX,Scale=1}
\fi
% Use upquote if available, for straight quotes in verbatim environments
\IfFileExists{upquote.sty}{\usepackage{upquote}}{}
\IfFileExists{microtype.sty}{% use microtype if available
  \usepackage[]{microtype}
  \UseMicrotypeSet[protrusion]{basicmath} % disable protrusion for tt fonts
}{}
\makeatletter
\@ifundefined{KOMAClassName}{% if non-KOMA class
  \IfFileExists{parskip.sty}{%
    \usepackage{parskip}
  }{% else
    \setlength{\parindent}{0pt}
    \setlength{\parskip}{6pt plus 2pt minus 1pt}}
}{% if KOMA class
  \KOMAoptions{parskip=half}}
\makeatother
\usepackage{xcolor}
\setlength{\emergencystretch}{3em} % prevent overfull lines
\setcounter{secnumdepth}{-\maxdimen} % remove section numbering
% Make \paragraph and \subparagraph free-standing
\ifx\paragraph\undefined\else
  \let\oldparagraph\paragraph
  \renewcommand{\paragraph}[1]{\oldparagraph{#1}\mbox{}}
\fi
\ifx\subparagraph\undefined\else
  \let\oldsubparagraph\subparagraph
  \renewcommand{\subparagraph}[1]{\oldsubparagraph{#1}\mbox{}}
\fi


\providecommand{\tightlist}{%
  \setlength{\itemsep}{0pt}\setlength{\parskip}{0pt}}\usepackage{longtable,booktabs,array}
\usepackage{calc} % for calculating minipage widths
% Correct order of tables after \paragraph or \subparagraph
\usepackage{etoolbox}
\makeatletter
\patchcmd\longtable{\par}{\if@noskipsec\mbox{}\fi\par}{}{}
\makeatother
% Allow footnotes in longtable head/foot
\IfFileExists{footnotehyper.sty}{\usepackage{footnotehyper}}{\usepackage{footnote}}
\makesavenoteenv{longtable}
\usepackage{graphicx}
\makeatletter
\def\maxwidth{\ifdim\Gin@nat@width>\linewidth\linewidth\else\Gin@nat@width\fi}
\def\maxheight{\ifdim\Gin@nat@height>\textheight\textheight\else\Gin@nat@height\fi}
\makeatother
% Scale images if necessary, so that they will not overflow the page
% margins by default, and it is still possible to overwrite the defaults
% using explicit options in \includegraphics[width, height, ...]{}
\setkeys{Gin}{width=\maxwidth,height=\maxheight,keepaspectratio}
% Set default figure placement to htbp
\makeatletter
\def\fps@figure{htbp}
\makeatother

\KOMAoption{captions}{tableheading}
\makeatletter
\makeatother
\makeatletter
\makeatother
\makeatletter
\@ifpackageloaded{caption}{}{\usepackage{caption}}
\AtBeginDocument{%
\ifdefined\contentsname
  \renewcommand*\contentsname{Table of contents}
\else
  \newcommand\contentsname{Table of contents}
\fi
\ifdefined\listfigurename
  \renewcommand*\listfigurename{List of Figures}
\else
  \newcommand\listfigurename{List of Figures}
\fi
\ifdefined\listtablename
  \renewcommand*\listtablename{List of Tables}
\else
  \newcommand\listtablename{List of Tables}
\fi
\ifdefined\figurename
  \renewcommand*\figurename{Figure}
\else
  \newcommand\figurename{Figure}
\fi
\ifdefined\tablename
  \renewcommand*\tablename{Table}
\else
  \newcommand\tablename{Table}
\fi
}
\@ifpackageloaded{float}{}{\usepackage{float}}
\floatstyle{ruled}
\@ifundefined{c@chapter}{\newfloat{codelisting}{h}{lop}}{\newfloat{codelisting}{h}{lop}[chapter]}
\floatname{codelisting}{Listing}
\newcommand*\listoflistings{\listof{codelisting}{List of Listings}}
\makeatother
\makeatletter
\@ifpackageloaded{caption}{}{\usepackage{caption}}
\@ifpackageloaded{subcaption}{}{\usepackage{subcaption}}
\makeatother
\makeatletter
\@ifpackageloaded{tcolorbox}{}{\usepackage[many]{tcolorbox}}
\makeatother
\makeatletter
\@ifundefined{shadecolor}{\definecolor{shadecolor}{rgb}{.97, .97, .97}}
\makeatother
\makeatletter
\makeatother
\ifLuaTeX
  \usepackage{selnolig}  % disable illegal ligatures
\fi
\IfFileExists{bookmark.sty}{\usepackage{bookmark}}{\usepackage{hyperref}}
\IfFileExists{xurl.sty}{\usepackage{xurl}}{} % add URL line breaks if available
\urlstyle{same} % disable monospaced font for URLs
\hypersetup{
  pdftitle={Assignment 2: Sequences, Limits, Derivatives, and Critical Points},
  colorlinks=true,
  linkcolor={blue},
  filecolor={Maroon},
  citecolor={Blue},
  urlcolor={Blue},
  pdfcreator={LaTeX via pandoc}}

\title{Assignment 2: Sequences, Limits, Derivatives, and Critical
Points}
\author{}
\date{}

\begin{document}
\maketitle
\ifdefined\Shaded\renewenvironment{Shaded}{\begin{tcolorbox}[frame hidden, enhanced, interior hidden, borderline west={3pt}{0pt}{shadecolor}, breakable, boxrule=0pt, sharp corners]}{\end{tcolorbox}}\fi

\hypertarget{assignment-qs}{%
\section{Assignment Qs}\label{assignment-qs}}

\begin{itemize}
\item
  Name
\item
  How long did this problem set take you?
\item
  How difficult was this problem set? very easy 1 2 3 4 5 very
  challenging

  \hypertarget{simplify-logarithms}{%
  \section{Simplify logarithms}\label{simplify-logarithms}}
\end{itemize}

Express each of the following as a single logarithm:\footnote{Inspired
  by Grimmer HW1.3. Assume log is natural log unless specified
  otherwise.}

\begin{enumerate}
\def\labelenumi{\alph{enumi}.}
\item
  \(\log(x) - 2\log(y) + \log(z)\)
\item
  \(2 \log(x) + log(1)\)
\item
  \(\log(2x) - 2\)
\end{enumerate}

\hypertarget{sequences}{%
\section{Sequences}\label{sequences}}

Write down the first three terms of each of the following sequences. In
each case, state whether the sequence is an arithmetric progression, a
geometric progression, or neither.\footnote{Pemberton and Rau 5.1.1}

\begin{enumerate}
\def\labelenumi{\alph{enumi}.}
\item
  \(u_n = 12 + n\)
\item
  \(u_n = n \times 3^n\)
\item
  \(u_n = 2^n\)
\end{enumerate}

\hypertarget{find-the-limit}{%
\section{Find the limit}\label{find-the-limit}}

In each of the following cases, state whether the sequence \(\{ u_n \}\)
tends to a limit, and find the limit if it exists:\footnote{Pemberton
  and Rau 5.1.3}

\begin{enumerate}
\def\labelenumi{\alph{enumi}.}
\item
  \(u_n = 1 + \frac{1}{12} n\)
\item
  \(u_n = \left( \frac{1}{12} \right)^n\)
\item
  \(\underset{x \to 2}\lim \frac{x^2 - 5x + 4}{x^3 - 3x -4}\)
\end{enumerate}

\hypertarget{determine-convergence-or-divergence}{%
\section{Determine convergence or
divergence}\label{determine-convergence-or-divergence}}

Determine whether each of the following sequences converges or diverges.
If it converges, find the limit.\footnote{Grimmer 2012 HW2.2}

\begin{enumerate}
\def\labelenumi{\alph{enumi}.}
\item
  \(a_n = \frac{3 + 5n^2}{n + n^2}\)
\item
  \(a_n = \frac{(-1)^{n-1} n}{n^2 + 1}\)
\end{enumerate}

\hypertarget{find-more-limits}{%
\section{Find more limits}\label{find-more-limits}}

Given that

\[
\underset{x \to a} \lim f(x) = -3, \quad \underset{x \to a} \lim g(x) = 0 , \quad \underset{x \to a} \lim h(x) = 8
\]

find the limits that exist. If the limit doesn't exist, explain
why.\footnote{Grimmer 2012 HW 2.4}

\begin{enumerate}
\def\labelenumi{\alph{enumi}.}
\item
  \(\underset{x \to a} \lim [f(x) + h(x)]\)
\item
  \(\underset{x \to a} \lim \frac{f(x)}{g(x)}\)
\end{enumerate}

\hypertarget{find-infinite-limits}{%
\section{Find infinite limits}\label{find-infinite-limits}}

Find the following infinite limits:\footnote{Gill 5.3 and 5.8}

\begin{quote}
Hint: use \textbf{L'Hôpital's Rule} to switch from
\(\underset{x\rightarrow \infty}{\lim} \left( \dfrac{f(x)}{g(x)} \right)\)
to
\(\underset{x\rightarrow \infty}{\lim} \left( \dfrac{f'(x)}{g'(x)} \right)\).
\end{quote}

\begin{enumerate}
\def\labelenumi{\alph{enumi}.}
\item
  \(\underset{x\rightarrow \infty}{\lim}\left[ \dfrac{9x^2}{x^2 +3} \right]\)
\item
  \(\underset{x\rightarrow \infty}{\lim} \left[ \dfrac{3^x}{x^3} \right]\)
\end{enumerate}

\hypertarget{assessing-continuity-and-differentiability}{%
\section{Assessing continuity and
differentiability}\label{assessing-continuity-and-differentiability}}

For each of the following functions, describe whether it is continuous
and/or differentiable at the point of transition of its two
formulas.\footnote{Simon and Blume 2.16}

\begin{enumerate}
\def\labelenumi{\alph{enumi}.}
\item
  \[f(x) = \begin{cases} 
     + x^2, & x \geq 0 \\
     - x^2, & x < 0
  \end{cases}\]
\item
  \[f(x) = \begin{cases} 
     x^3, & x \leq 1 \\
     x, & x > 1
  \end{cases}\]
\end{enumerate}

\hypertarget{possible-derivative}{%
\section{Possible derivative}\label{possible-derivative}}

A friend shows you this graph of a function \(f(x)\):\footnote{Grimmer
  HW3.6}

\begin{figure}

{\centering \includegraphics[width=0.9\textwidth,height=\textheight]{02-seq-limits-critical-points_files/figure-pdf/polynomial-1.pdf}

}

\end{figure}

\newpage{}

Which of the following could be a graph of \(f'(x)\)? For each graph,
explain why or why not it might be the derivative of \(f(x)\).

\begin{figure}

{\centering \includegraphics[width=0.9\textwidth,height=\textheight]{02-seq-limits-critical-points_files/figure-pdf/derivatives-1.pdf}

}

\end{figure}

\newpage{}

What if the figure below was the graph of \(f(x)\)? Which of the graphs
might potentially be the derivative of \(f(x)\) then?

\begin{figure}

{\centering \includegraphics[width=0.9\textwidth,height=\textheight]{02-seq-limits-critical-points_files/figure-pdf/straight-line-1.pdf}

}

\end{figure}

\newpage{}

\hypertarget{calculate-derivatives}{%
\section{Calculate derivatives}\label{calculate-derivatives}}

Differentiate the following functions:\footnote{Grimmer HW2.3}

\begin{enumerate}
\def\labelenumi{\alph{enumi}.}
\item
  \(f(x) = 4x^3 + 2x^2 + 5x + 11\)
\item
  \(y = \sqrt{30}\)
\item
  \(h(t) = \log(9t+1)\)
\item
  \(f(x) = \log(x^2e^x)\)
\item
  \(h(y) = \left( \dfrac{1}{y^2} - \dfrac{3}{y^4} \right) (y+5y^3)\)
\item
  \(h(x) = \frac{x}{\log(x)}\)
\end{enumerate}

\hypertarget{use-the-product-and-quotient-rules}{%
\section{Use the product and quotient
rules}\label{use-the-product-and-quotient-rules}}

Differentiate the following function twice -- once using the product
rule, and once using the quotient rule:\footnote{Grimmer HW2.4}

\[f(x) = \dfrac{x^2-2x}{x^4 + 6}\]

\hypertarget{composite-functions}{%
\section{Composite functions}\label{composite-functions}}

For each of the following pairs of functions \(g(x)\) and \(h(z)\),
write out the composite function \(g(h[z])\) and \(h(g[x])\). Describe
the domain of the composite function.\footnote{Simon and Blume 4.1}

\begin{enumerate}
\def\labelenumi{\alph{enumi}.}
\item
  \(g(x) = x^3, \quad h(z) = (z - 1)(z + 1)\)
\end{enumerate}

\hypertarget{chain-rule}{%
\section{Chain rule}\label{chain-rule}}

Use the chain rule to compute the derivative of the composite functions
in the previous section from the derivatives of the two component
functions. Then, compute each derivative directly using your expression
for the composite function. Simplify and compare your
answers.\footnote{Simon and Blume 4.3}

\begin{enumerate}
\def\labelenumi{\alph{enumi}.}
\item
  \(g(x) = x^3, \quad h(z) = (z - 1)(z + 1)\)
\item
  \(g(x) = 4x + 2, \quad h(z) = \frac{1}{4}(z - 2)\)
\end{enumerate}

\hypertarget{ai-resources-statement}{%
\section{AI / Resources statement}\label{ai-resources-statement}}

Please list (in detail) all resources you used for this assignment. If
you worked with people, list them here as well. It is not enough to say
that you used a resource for help, you need to be specific on the link
and \emph{how} it was helpful. W/R/T gen AI tools (including GPT, etc. )
you cannot use them to do work on your behalf -- you cannot put in any
of the questions, etc. You can ask for help on logic / sample problems.
If you do use GPT or other AI tools, you need to provide a link to your
chat transcript. Any suspected academic integrity violations will be
immediately reported.



\end{document}
