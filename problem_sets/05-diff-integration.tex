% Options for packages loaded elsewhere
\PassOptionsToPackage{unicode}{hyperref}
\PassOptionsToPackage{hyphens}{url}
\documentclass[
]{article}
\usepackage{xcolor}
\usepackage[margin=1in]{geometry}
\usepackage{amsmath,amssymb}
\setcounter{secnumdepth}{-\maxdimen} % remove section numbering
\usepackage{iftex}
\ifPDFTeX
  \usepackage[T1]{fontenc}
  \usepackage[utf8]{inputenc}
  \usepackage{textcomp} % provide euro and other symbols
\else % if luatex or xetex
  \usepackage{unicode-math} % this also loads fontspec
  \defaultfontfeatures{Scale=MatchLowercase}
  \defaultfontfeatures[\rmfamily]{Ligatures=TeX,Scale=1}
\fi
\usepackage{lmodern}
\ifPDFTeX\else
  % xetex/luatex font selection
\fi
% Use upquote if available, for straight quotes in verbatim environments
\IfFileExists{upquote.sty}{\usepackage{upquote}}{}
\IfFileExists{microtype.sty}{% use microtype if available
  \usepackage[]{microtype}
  \UseMicrotypeSet[protrusion]{basicmath} % disable protrusion for tt fonts
}{}
\makeatletter
\@ifundefined{KOMAClassName}{% if non-KOMA class
  \IfFileExists{parskip.sty}{%
    \usepackage{parskip}
  }{% else
    \setlength{\parindent}{0pt}
    \setlength{\parskip}{6pt plus 2pt minus 1pt}}
}{% if KOMA class
  \KOMAoptions{parskip=half}}
\makeatother
\usepackage{graphicx}
\makeatletter
\newsavebox\pandoc@box
\newcommand*\pandocbounded[1]{% scales image to fit in text height/width
  \sbox\pandoc@box{#1}%
  \Gscale@div\@tempa{\textheight}{\dimexpr\ht\pandoc@box+\dp\pandoc@box\relax}%
  \Gscale@div\@tempb{\linewidth}{\wd\pandoc@box}%
  \ifdim\@tempb\p@<\@tempa\p@\let\@tempa\@tempb\fi% select the smaller of both
  \ifdim\@tempa\p@<\p@\scalebox{\@tempa}{\usebox\pandoc@box}%
  \else\usebox{\pandoc@box}%
  \fi%
}
% Set default figure placement to htbp
\def\fps@figure{htbp}
\makeatother
\setlength{\emergencystretch}{3em} % prevent overfull lines
\providecommand{\tightlist}{%
  \setlength{\itemsep}{0pt}\setlength{\parskip}{0pt}}
\usepackage{bookmark}
\IfFileExists{xurl.sty}{\usepackage{xurl}}{} % add URL line breaks if available
\urlstyle{same}
\hypersetup{
  pdftitle={PSET 5: Functions of several variables and optimization with several variables},
  hidelinks,
  pdfcreator={LaTeX via pandoc}}

\title{PSET 5: Functions of several variables and optimization with
several variables}
\author{}
\date{\vspace{-2.5em}}

\begin{document}
\maketitle

\textbf{Note: all homework uploads should be as a PDF \emph{and} have
the questions identified}. We'll be giving zero credit for submissions
that don't follow this protocol as it adds considerable time to grading.
Thank you!

\begin{itemize}
\item
  Name
\item
  How long did this problem set take you?
\item
  How difficult was this problem set? very easy 1 2 3 4 5 very
  challenging
\end{itemize}

\section{Find first partial
derivatives}\label{find-first-partial-derivatives}

Find all of the first partial derivatives of each function.\footnote{Grimmer
  HW6.3}

\begin{enumerate}
\def\labelenumi{\alph{enumi}.}
\item
  \(f(x,y) = 3x - 2y^4\)

  ~ ~\\
  \strut \\
\item
  \(f(x,y) = x^5 + 3x^3y^2 + 3xy^4\)

  ~ ~\\
  \strut \\
\item
  \(g(x,y) = xe^{3y}\)

  ~ ~\\
  \strut \\
\item
  \(k(x,y) = \frac{x-y}{x+y}\)

  ~\\
\item
  \(h(x,y,z) = x^2 e^{yz}\)

  ~\\
  \newpage
\end{enumerate}

\section{Find the gradient}\label{find-the-gradient}

Find the gradient \(\nabla f\) of the following functions and evaluate
them at the given points.\footnote{Grimmer HW6.4}

\begin{enumerate}
\def\labelenumi{\alph{enumi}.}
\item
  \(f(x,y) = \sqrt{x^2 + y^2}\), \quad \((x,y) = (3,4)\)

  ~\\
  \strut \\
\item
  \(f(x,y,z) = (x+z)e^{x-y}\), \quad \((x,y,z) = (1,1,1)\)

  ~\\
  \strut \\
\end{enumerate}

\section{Find the Hessian}\label{find-the-hessian}

Find the Hessian \(H\) for the following functions.\footnote{Grimmer
  HW7.3}

\begin{enumerate}
\def\labelenumi{\alph{enumi}.}
\item
  \(g(x,y) = x^4 - 3x^2 y^3\)

  \strut \\
  \strut ~\\
  \strut ~ ~\\
  \strut ~
\item
  \(f(x,y,z) = xyz - x^2\)

  \strut \\
  \strut ~\\
  \strut ~ ~\\
  \strut \\
\end{enumerate}

\section{Find the critical points}\label{find-the-critical-points}

Find the local minimum values, local maximum values, and saddle point(s)
of the function. Remember the process we discussed in class: Calculate
the gradient, set it equal to zero to solve the system of equations,
calculate the Hessian, and assess the Hessian at critical values. Be
sure to show your work on each of these steps.\footnote{Grimmer HW7.4}

\begin{enumerate}
\def\labelenumi{\alph{enumi}.}
\item
  \(f(x,y) = x^4 + y^4 - 4xy + 2\)

  ~ ~ ~\\
\item
  \(k(x,y) = (1+xy)(x+y)\)

  ~\\
  \newpage
\end{enumerate}

\section{Definite integrals}\label{definite-integrals}

Solve the following definite integrals using the antiderivative
method.\footnote{Gill 5.10 and Grimmer HW4.1}

For all these problems, the basic approach to compute the definite
integral of \(f(x)\) from \(a\) to \(b\) is by using the formula
\(F(b) - F(a)\), where \(F(x)\) is the \textbf{antiderivative} of \(f\).

\begin{enumerate}
\def\labelenumi{\alph{enumi}.}
\item
  \(\int_6^8 x^3 \,dx\)

  ~\\
\item
  \(\int_{-1}^{0} (3x^2 -1) \,dx\)

  ~\\
\item
  \(\int_0^1 x^{\frac{3}{7}} \,dx\)

  ~\\
\item
  \(\int_1^2 \dfrac{1}{t^2} \,dt\)

  ~\\
\item
  \(\int_2^4 e^y \,dy\)

  ~\\
\item
  \(\int_8^9 2^x \,dx\)

  ~\\
\item
  \(\int_3^3 \sqrt{x^5 + 2} \,dx\)

  ~\\
  \newpage
\end{enumerate}

\section{Applied integration}\label{applied-integration}

A group of three unidentified first-year graduate students at the
University of Chicago are worn out after a week of math camp. Wanting to
unwind, the students agree to not talk about math and decide to chat
over some casual drinks at Medici.

After five shots of tequila each, two pitchers of beer, a bottle of
wine, and a large Chicago-style pizza, the three students have had
enough fun and decide to start the trip back home.

\begin{itemize}
\tightlist
\item
  Student \(A\) gets on a bike and starts pedaling away at a velocity of
  \(v_A(t) = 2t^4 + t\), where \(t\) represents minutes. However, the
  student crashes into the side of an Uber and ends the journey after
  only 2 minutes.
\item
  Student \(B\) has no bike, so starts running at a velocity of
  \(v_B(t) = 4\sqrt{t}\). Sadly, after only 4 minutes, the student's
  legs give out and the student decides to sing a song, instead.
\item
  Student \(C\) can't even stand up, so has no choice but to slowly
  crawl at a velocity of \(v_C(t) = 2e^{-t}\). Student \(C\) steadily
  plods along for 20 minutes before falling asleep on the sidewalk.
\end{itemize}

Generally, if an object moves along a straight line with position
function \(s(t)\), then its velocity is \(v(t) = s'(t)\). The
Fundamental Theorem of Calculus then tells us that

\[
\begin{aligned}
\textrm{Total distance traveled} &= \int_{t_1}^{t_2} v(t) \,dt\\
s(t_2) - s(t_1) &= \int_{t_1}^{t_2} v(t) \,dt
\end{aligned}
\]

Without using a calculator, use this formula to find the distance
traveled by Students \(A\), \(B\), and \(C\). (Assume, however
unrealistic in may be, that all three students traveled in a straight
line.) Who traveled the farthest? The least far?\footnote{Grimmer HW4.2}

~\\
\newpage

\section{Indefinite integrals}\label{indefinite-integrals}

Calculate the following indefinite integrals:\footnote{Gill 5.13 and
  5.14}

\begin{enumerate}
\def\labelenumi{\alph{enumi}.}
\item
  \(\int (x^2-x^{-\frac{1}{2}}) \,dx\)

  ~\\
\item
  \(\int 360t^6 \,dt\)

  ~\\
\item
  \(\int 2x\log(x^2) \,dx\)

  ~\\
\end{enumerate}

\section{Determining convergence}\label{determining-convergence}

Determine whether each integral is convergent or divergent. Evaluate
those that are convergent.\footnote{Grimmer HW 4.3}

\begin{enumerate}
\def\labelenumi{\alph{enumi}.}
\item
  \(\int_1^{\infty} \left(\frac{1}{3x}\right)^2 \,dx\)

  ~\\
\item
  \(\int_0^{\infty} \cos (x) \, dx\)

  \strut \\
  \strut ~ ~ ~ ~
\item
  \(\int_0^{\infty} e^{-x} \,dx\)

  \strut \\
  \strut ~ ~ ~ ~
\item
  \(\int_{-\infty}^0 x^3 \,dx\)

  \strut \\
  \strut ~ ~ ~ ~
\end{enumerate}

\section{More integrals}\label{more-integrals}

Calculate the following integrals:\footnote{Grimmer HW7.5}

\begin{enumerate}
\def\labelenumi{\alph{enumi}.}
\item
  \(\int_{0}^1 \int_{2}^{3} x^2y^3 \,\, dxdy\)

  \strut \\
  \strut ~ ~ ~\\
  \strut \\
\item
  \(\int_{2}^{3} \int_{0}^1 x^2y^3 \,\, dy dx\)

  \textless!--

  \[
   \begin{aligned}
    \int_{2}^3 \int_{0}^{1} x^2y^3 dydx &= \int_2^3 [\frac{1}{4} x^2y^4 |^{y=1}_{y=0}]dx\\
    &= \int_{2}^{3} [\frac{1}{4} 1^4 x^2 - \frac{1}{4} 0^4 x^2 ] dx = \int_{2}^{3} \frac{1}{4} x^2 dx\\
    &= \frac{1}{12}x^3 |_{x=2}^{x=3} = \frac{19}{12}
   \end{aligned}
   \]

  A and B demonstrate the order of integration doesn't matter, you
  should get the same answer either way.
\end{enumerate}

\#\# AI and Resources statement * Please list (in detail) all resources
you used for this assignment. If you worked with people, list them here
as well. It is not enough to say that you used a resource for help, you
need to be specific on the link and \emph{how} it was helpful. W/R/T gen
AI tools (including GPT, etc. ) you cannot use them to do work on your
behalf -- you cannot put in any of the questions, etc. You can ask for
help on logic / sample problems. If you do use GPT or other AI tools,
you need to provide a link to your chat transcript. Any suspected
academic integrity violations will be immediately reported.

\begin{verbatim}
    -->      
    \     \  \     
    \ 
\end{verbatim}

\begin{enumerate}
\def\labelenumi{\alph{enumi}.}
\item
  \(\int_{0}^1 \int_0^{\sqrt{1-x^2}} 2x^3y \, \, dy dx\)

  \strut \\
  \strut ~ ~ ~\\
  \strut ~
\end{enumerate}

\end{document}
