% Options for packages loaded elsewhere
\PassOptionsToPackage{unicode}{hyperref}
\PassOptionsToPackage{hyphens}{url}
\documentclass[
]{article}
\usepackage{xcolor}
\usepackage[margin=1in]{geometry}
\usepackage{amsmath,amssymb}
\setcounter{secnumdepth}{-\maxdimen} % remove section numbering
\usepackage{iftex}
\ifPDFTeX
  \usepackage[T1]{fontenc}
  \usepackage[utf8]{inputenc}
  \usepackage{textcomp} % provide euro and other symbols
\else % if luatex or xetex
  \usepackage{unicode-math} % this also loads fontspec
  \defaultfontfeatures{Scale=MatchLowercase}
  \defaultfontfeatures[\rmfamily]{Ligatures=TeX,Scale=1}
\fi
\usepackage{lmodern}
\ifPDFTeX\else
  % xetex/luatex font selection
\fi
% Use upquote if available, for straight quotes in verbatim environments
\IfFileExists{upquote.sty}{\usepackage{upquote}}{}
\IfFileExists{microtype.sty}{% use microtype if available
  \usepackage[]{microtype}
  \UseMicrotypeSet[protrusion]{basicmath} % disable protrusion for tt fonts
}{}
\makeatletter
\@ifundefined{KOMAClassName}{% if non-KOMA class
  \IfFileExists{parskip.sty}{%
    \usepackage{parskip}
  }{% else
    \setlength{\parindent}{0pt}
    \setlength{\parskip}{6pt plus 2pt minus 1pt}}
}{% if KOMA class
  \KOMAoptions{parskip=half}}
\makeatother
\usepackage{graphicx}
\makeatletter
\newsavebox\pandoc@box
\newcommand*\pandocbounded[1]{% scales image to fit in text height/width
  \sbox\pandoc@box{#1}%
  \Gscale@div\@tempa{\textheight}{\dimexpr\ht\pandoc@box+\dp\pandoc@box\relax}%
  \Gscale@div\@tempb{\linewidth}{\wd\pandoc@box}%
  \ifdim\@tempb\p@<\@tempa\p@\let\@tempa\@tempb\fi% select the smaller of both
  \ifdim\@tempa\p@<\p@\scalebox{\@tempa}{\usebox\pandoc@box}%
  \else\usebox{\pandoc@box}%
  \fi%
}
% Set default figure placement to htbp
\def\fps@figure{htbp}
\makeatother
\setlength{\emergencystretch}{3em} % prevent overfull lines
\providecommand{\tightlist}{%
  \setlength{\itemsep}{0pt}\setlength{\parskip}{0pt}}
\usepackage{bookmark}
\IfFileExists{xurl.sty}{\usepackage{xurl}}{} % add URL line breaks if available
\urlstyle{same}
\hypersetup{
  pdftitle={Linear algebra},
  hidelinks,
  pdfcreator={LaTeX via pandoc}}

\title{Linear algebra}
\author{}
\date{\vspace{-2.5em}}

\begin{document}
\maketitle

\begin{itemize}
\item
  Name
\item
  How long did this problem set take you?
\item
  How difficult was this problem set? very easy 1 2 3 4 5 very
  challenging
\end{itemize}

\section{Matrix arithmetic}\label{matrix-arithmetic}

Suppose

\[\mathbf{x} = \begin{bmatrix}
    3 \\
    2q \\
    4
\end{bmatrix} \quad \mbox{and} \quad \mathbf{y} = \begin{bmatrix}
    p + 2 \\
    -5 \\
    3r
\end{bmatrix}\].

If \(\mathbf{x} = 2 \mathbf{y}\), find \(p, q, r\).\footnote{inspired by
  Pemberton and Rau 11.1.3}

\section{Check for linear dependence}\label{check-for-linear-dependence}

Which of the following sets of vectors are linearly
dependent?\footnote{inspired by Pemberton and Rau 11.1.4}

In each part, you can denote each vector as
\(\mathbf{a}, \mathbf{b}, \mathbf{c}\) respectively.

\begin{enumerate}
\def\labelenumi{\alph{enumi}.}
\item
  \(\begin{bmatrix} 1 \\ 2 \\ 3 \end{bmatrix}, \begin{bmatrix} 4 \\ 5 \\ 6 \end{bmatrix}, \begin{bmatrix} 7 \\ 8 \\ 9 \end{bmatrix}\)
\item
  \(\begin{bmatrix} 13 \\ 7 \\ 9 \\ 2 \end{bmatrix}, \begin{bmatrix} 0 \\ 0 \\ 0 \\ 0 \end{bmatrix}, \begin{bmatrix} 3 \\ -2 \\ 5 \\ 8 \end{bmatrix}\)
\item
  \(\begin{bmatrix} 1 \\ 2 \\ 1 \end{bmatrix}, \begin{bmatrix} 2 \\ -2 \\ -1 \end{bmatrix}, \begin{bmatrix} 2 \\ 1 \\ 3 \end{bmatrix}\)
\end{enumerate}

\section{Vector length}\label{vector-length}

Find the length of the following vectors:\footnote{inspired by Simon and
  Blume 10.10}

\begin{enumerate}
\def\labelenumi{\alph{enumi}.}
\item
  \((4, 2)\)
\item
  \((2,2,2)\)
\item
  \((4, 2, 3, 1)\)
\item
  \((0, 0, 0, 0, 3)\)
\end{enumerate}

\section{Law of cosines}\label{law-of-cosines}

The \textbf{law of cosines} states:

\[\cos(\theta) = \frac{\mathbf{v} \cdot \mathbf{w}}{\|\mathbf{v}\| \: \|\mathbf{w}\|}\]

where \(\theta\) is the angle from \(\mathbf{w}\) to \(\mathbf{v}\)
measured in radians. Of importance, \(\arccos()\) is the inverse of
\(\cos()\):

\[\theta = \arccos \left( \frac{\mathbf{v} \cdot \mathbf{w}}{\|\mathbf{v}\| \: \|\mathbf{w}\|} \right)\]

For each of the following pairs of vectors, calculate the angle between
them. Report your answers in both radians and degrees. To convert
between radians and degrees:\footnote{inspired by Simon and Blume 10.12}

\[\text{Degrees} = \text{Radians} \times \dfrac{180^{o}}{\pi}\]

\begin{enumerate}
\def\labelenumi{\alph{enumi}.}
\item
  \(\mathbf{v} = (4, 1), \quad \mathbf{w} = (2, 8)\)
\item
  \(\mathbf{v} = (1, 1, 0), \quad \mathbf{w} = (2,1, 2)\)
\end{enumerate}

\section{Matrix algebra}\label{matrix-algebra}

Using the matrices below, calculate the following. Some may not be
defined; if that is the case, say so.\footnote{inspired by Grimmer HW5.3}

\[
\mathbf{A} = \left[
  \begin{array}{r}
    3 \\
    2 \\
    9
  \end{array}
\right]
\quad
\mathbf{B} = \left[
  \begin{array}{r}
    8 \\
    0 \\
    -1
  \end{array}
\right]
\quad
\mathbf{C} = \left[
  \begin{array}{rrr}
    7 & -1 & 5 \\
    0 & 2 & -4
  \end{array}
\right]
\quad
\mathbf{D} = \left[
  \begin{array}{rr}
    3 & 1 \\
    3 & 4 \\
    3 & -7
  \end{array}
\right]
\quad
\mathbf{E} = \left[
  \begin{array}{rrr}
    5 & 2 & 3 \\
    1 & 0 & -4 \\
    -2 & 1 & -6
  \end{array}
\right]
\]

\[
\mathbf{F} = \left[
  \begin{array}{rrr}
    4 & 1 & -5 \\
    0 & 7 & 7 \\
    2 & -3 & 0
  \end{array}
\right]
\quad
\mathbf{G} = \left[
  \begin{array}{rrr}
    2 & -8 & -5 \\
    -3 & 7 & -4 \\
    1 & 0 & 3 \\
    1 & 2 & 6
  \end{array}
\right]
\quad
\mathbf{K} = \left[
  \begin{array}{rrrr}
    9 & 
    -2 &
    -1 &
    0
  \end{array}
\right]
\]

\[
\mathbf{L} = \left[
  \begin{array}{r}
    5 \\ 0 \\ 3 \\ 1
  \end{array}
\right]
\quad
\mathbf{M} = \begin{bmatrix}
-1 & 1 \\
-1 & 3
\end{bmatrix}
\]

\begin{enumerate}
\def\labelenumi{\alph{enumi}.}
\item
  \(\mathbf{A} + \mathbf{B}\)
\item
  \(-\mathbf{G}\)
\item
  \(\mathbf{D}'\)
\item
  \(\mathbf{C} + \mathbf{D}\)
\item
  \(\mathbf{A}' \mathbf{B}\)
\item
  \(\mathbf{BC}\)
\item
  \(\mathbf{FB}\)
\item
  \(\mathbf{E} - 5\mathbf{I_3}\)
\item
  \(\mathbf{M}^2\)
\end{enumerate}

\section{Matrix inversion}\label{matrix-inversion}

Invert each of the following matricies by hand (you can use a calculator
or computer to check your solution, but be sure to show your work).
Verify you have the correct inverse by calculating
\(\mathbf{XX}^{-1} = \mathbf{I}\). Not all of the matricies may be
invertible - if not, show why.\footnote{inspired by Simon and Blume 8.19}

\begin{enumerate}
\def\labelenumi{\alph{enumi}.}
\item
  \(\left[ \begin{array}{rr} 2 & 1 \\ 1 & 1 \end{array}\right]\)
\item
  \(\left[ \begin{array}{rr} 2 & 1 \\ 4 & -2 \end{array}\right]\)
\item
  \(\left[ \begin{array}{rrr} 2 & 4 & 0 \\ 4 & 6 & 3 \\ -6 & -10 & 0 \end{array}\right]\)
\end{enumerate}

\section{Dummy encoding for categorical
variables}\label{dummy-encoding-for-categorical-variables}

Ordinary least squares regression is a common method for obtaining
regression parameters relating a set of explanatory variables with a
continuous outcome of interest. The vector \(\hat{\mathbf{b}}\) that
contains the intercept and the regression slope is calculated by the
equation:

\[\hat{\mathbf{b}} = (\mathbf{X'X})^{-1}\mathbf{X'y}\]

If an explanatory variable is nominal (i.e.~ordering does not matter)
with more than two classes
(e.g.~\(\{\text{White}, \text{Black}, \text{Asian}, \text{Mixed}, \text{Other}\}\)),
the variable must be modified to include in the regression model. A
common technique known as \textbf{dummy encoding} converts the column
into a series of \(n-1\) binary (\(0/1\)) columns where each column
represents a single class and \(n\) is the total number of unique
classes in the original column. Explain why this method converts the
column into \(n-1\) columns, rather than \(n\) columns, in terms of
linear algebra. \textbf{Reminder: \(\mathbf{X}\) contains both the dummy
encoded columns as well as a column of \(1\)s representing the
intercept.}\footnote{inspired by Benjamin Soltoff}

\section{Solve the system of
equations}\label{solve-the-system-of-equations}

Solve the following systems of equations for \(x, y, z\), either via
matrix inversion or substitution:\footnote{inspired by Gill 4.19}

\begin{enumerate}
\def\labelenumi{\alph{enumi}.}
\item
  System \#1

  \[
   \begin{aligned}
       x +   y + 2z &=  2 \\
       3x -  2y -  z &= 1 \\
       y -  z &= 3
   \end{aligned}
   \]
\item
  System \#2

  \[
   \begin{aligned}
   x - y + 2z &= 2 \\
   4x + y -2z &= 10 \\
   x + 3y +z &= 0
   \end{aligned}
   \]
\end{enumerate}

\section{\texorpdfstring{Multiplying by
\(\mathbf{0}\)}{Multiplying by \textbackslash mathbf\{0\}}}\label{multiplying-by-mathbf0}

When it comes to real numbers, we know that if \(xy = 0\), then either
\(x=0\) or \(y=0\) or both. One might believe that a similar idea
applies to matricies, but one would be wrong. Prove that if the matrix
product \(\mathbf{AB=0}\) (by which we mean a matrix of appropriate
dimensionality made up entirely of zeroes), then it is not necessarily
true that either \(\mathbf{A=0}\) or \(\mathbf{B=0}\). Hint: in order to
prove that something is not always true, simply identify one example
where \(\mathbf{AB=0}, \: \mathbf{A, B \neq 0}\).\footnote{inspired by
  Grimmer HW5.5}

\subsection{AI and Resources
statement}\label{ai-and-resources-statement}

\begin{itemize}
\tightlist
\item
  Please list (in detail) all resources you used for this assignment. If
  you worked with people, list them here as well. It is not enough to
  say that you used a resource for help, you need to be specific on the
  link and \emph{how} it was helpful. W/R/T gen AI tools (including GPT,
  etc. ) you cannot use them to do work on your behalf -- you cannot put
  in any of the questions, etc. You can ask for help on logic / sample
  problems. If you do use GPT or other AI tools, you need to provide a
  link to your chat transcript. Any suspected academic integrity
  violations will be immediately reported.
\end{itemize}

\end{document}
