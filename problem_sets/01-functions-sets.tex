% Options for packages loaded elsewhere
\PassOptionsToPackage{unicode}{hyperref}
\PassOptionsToPackage{hyphens}{url}
\documentclass[
]{article}
\usepackage{xcolor}
\usepackage[margin=1in]{geometry}
\usepackage{amsmath,amssymb}
\setcounter{secnumdepth}{-\maxdimen} % remove section numbering
\usepackage{iftex}
\ifPDFTeX
  \usepackage[T1]{fontenc}
  \usepackage[utf8]{inputenc}
  \usepackage{textcomp} % provide euro and other symbols
\else % if luatex or xetex
  \usepackage{unicode-math} % this also loads fontspec
  \defaultfontfeatures{Scale=MatchLowercase}
  \defaultfontfeatures[\rmfamily]{Ligatures=TeX,Scale=1}
\fi
\usepackage{lmodern}
\ifPDFTeX\else
  % xetex/luatex font selection
\fi
% Use upquote if available, for straight quotes in verbatim environments
\IfFileExists{upquote.sty}{\usepackage{upquote}}{}
\IfFileExists{microtype.sty}{% use microtype if available
  \usepackage[]{microtype}
  \UseMicrotypeSet[protrusion]{basicmath} % disable protrusion for tt fonts
}{}
\makeatletter
\@ifundefined{KOMAClassName}{% if non-KOMA class
  \IfFileExists{parskip.sty}{%
    \usepackage{parskip}
  }{% else
    \setlength{\parindent}{0pt}
    \setlength{\parskip}{6pt plus 2pt minus 1pt}}
}{% if KOMA class
  \KOMAoptions{parskip=half}}
\makeatother
\usepackage{graphicx}
\makeatletter
\newsavebox\pandoc@box
\newcommand*\pandocbounded[1]{% scales image to fit in text height/width
  \sbox\pandoc@box{#1}%
  \Gscale@div\@tempa{\textheight}{\dimexpr\ht\pandoc@box+\dp\pandoc@box\relax}%
  \Gscale@div\@tempb{\linewidth}{\wd\pandoc@box}%
  \ifdim\@tempb\p@<\@tempa\p@\let\@tempa\@tempb\fi% select the smaller of both
  \ifdim\@tempa\p@<\p@\scalebox{\@tempa}{\usebox\pandoc@box}%
  \else\usebox{\pandoc@box}%
  \fi%
}
% Set default figure placement to htbp
\def\fps@figure{htbp}
\makeatother
\setlength{\emergencystretch}{3em} % prevent overfull lines
\providecommand{\tightlist}{%
  \setlength{\itemsep}{0pt}\setlength{\parskip}{0pt}}
\usepackage{bookmark}
\IfFileExists{xurl.sty}{\usepackage{xurl}}{} % add URL line breaks if available
\urlstyle{same}
\hypersetup{
  pdftitle={Linear equations, inequalities, sets and functions, quadratics},
  hidelinks,
  pdfcreator={LaTeX via pandoc}}

\title{Linear equations, inequalities, sets and functions, quadratics}
\author{}
\date{\vspace{-2.5em}}

\begin{document}
\maketitle

\section{Simplify expressions}\label{simplify-expressions}

\begin{enumerate}
\def\labelenumi{\arabic{enumi}.}
\tightlist
\item
  Simplify the following expressions as much as possible:\footnote{inspired
    by Gill 1.1}
\end{enumerate}

\begin{enumerate}
\def\labelenumi{\alph{enumi}.}
\item
  \((-x^2y^3)^3\)
\item
  \(9(3^0)\)
\item
  \((3^2a^2)^2(6a^4)\)
\item
  \(\Big(\dfrac{x^3}{x^4}\Big)^3\)
\item
  \((-2)^{(4-9)}\)
\item
  \(\left(\dfrac{1}{27b^4}\right)^{1/3}\)
\item
  \(y^5y^6y^5y^2\)
\item
  \(\dfrac{13a/7b}{13b/2a}\)
\end{enumerate}

\section{Simplify a (more complex)
expression}\label{simplify-a-more-complex-expression}

\begin{enumerate}
\def\labelenumi{\arabic{enumi}.}
\setcounter{enumi}{1}
\tightlist
\item
  Simplify the following expression:\footnote{inspired by Gill 1.2}
\end{enumerate}

\[(a+b)^2 + (a-b)^2 + 2(a+b)(a-b) - 3a^2\]

\section{Graph sketching}\label{graph-sketching}

\begin{enumerate}
\def\labelenumi{\arabic{enumi}.}
\setcounter{enumi}{2}
\tightlist
\item
  Let the functions \(f(x)\) and \(g(x)\) be defined for all
  \(x \in \Re\) by
\end{enumerate}

\[
f(x) = \left\{
    \begin{array}{ll}
        | x -2 | & \quad \mbox{if} \ x < 1 \\
        1 & \quad \mbox{if} \ x \geq 1
    \end{array}
\right., \quad
g(x) = \left\{
    \begin{array}{ll}
        x^3 & \quad \mbox{if} \ x < 2 \\
        4 & \quad \mbox{if} \ x \geq 2
    \end{array}
\right.
\]

Sketch the graphs of:\footnote{inspired by Pemberton and Rau problem 3-1}

\begin{enumerate}
\def\labelenumi{\arabic{enumi}.}
\item
  \(y = f(x)\)
\item
  \(y = g(x)\)
\item
  \(y = f(g(x))\)
\item
  \(y = g(f(x))\)
\end{enumerate}

\section{Root finding}\label{root-finding}

\begin{enumerate}
\def\labelenumi{\arabic{enumi}.}
\setcounter{enumi}{3}
\tightlist
\item
  Find the roots (solutions) to the following quadratic
  equations.\footnote{Gill 1.25}
\end{enumerate}

\[x = \dfrac{-b \pm \sqrt{b^2 - 4ac}}{2a}\]

\begin{enumerate}
\def\labelenumi{\alph{enumi}.}
\item
  \(9x^2 - 3x - 12 = 0\)
\item
  \(x^2 - 2x - 16 = 0\)
\item
  \(6x^2 - 6x - 6 = 0\)
\end{enumerate}

\section{Systems of linear equations}\label{systems-of-linear-equations}

\begin{enumerate}
\def\labelenumi{\arabic{enumi}.}
\setcounter{enumi}{4}
\tightlist
\item
  Solve the following systems of equations for their unknown values. If
  there is no solution, indicate as such.
\end{enumerate}

\begin{enumerate}
\def\labelenumi{\alph{enumi}.}
\item
  Two unknowns\footnote{inspired by OpenStax Algebra ex 7.1.12}

  \[
   \begin{aligned}
   3x - 2y &= 17 \\
   5x - 10y &= -10
   \end{aligned}
   \]
\item
  Three unknowns\footnote{inspired by OpenStax Algebra ex 7.2.12}

  \[
   \begin{aligned}
   5x - 2y + 3z &= 9 \\
   2x - 4y - 3z &= -9 \\
   x + 6y - 8z &= 24
   \end{aligned}
   \]
\item
  An animal shelter has a total of 124 animals comprised of cats, dogs,
  and rabbits. If the number of rabbits is 4 less than twice the number
  of dogs, and there are 76 more cats than dogs, how many of each animal
  are at the shelter?\footnote{inspired by OpenStax Algebra 7.2.54}
\end{enumerate}

\section{Work with sets}\label{work-with-sets}

\begin{enumerate}
\def\labelenumi{\arabic{enumi}.}
\setcounter{enumi}{5}
\tightlist
\item
  Using the sets
\end{enumerate}

\[
\begin{aligned}
A&=\left\{2,3,7,9,12,13 \right\} \\
B&=\left\{ x: 6\leq x \leq 12 \  \mbox{and} \ x \ \mbox{is an even integer} \right\} \\
C&=\left\{ x: 2< x < 25 \  \mbox{and} \  x \  \mbox{is prime}  \right\} \\
D&=\left\{ 1,4,9,16,25, \ldots  \right\} \\
\end{aligned}
\]

identify the following:\footnote{inspired by Grimmer HW1.1}

\begin{enumerate}
\def\labelenumi{\arabic{enumi}.}
\item
  \(A\cup B\)
\item
  (\(A\cup B) \cap C\)
\item
  \(C \cap D\)
\end{enumerate}

\subsection{AI and Resources
statement}\label{ai-and-resources-statement}

\begin{itemize}
\tightlist
\item
  Please list (in detail) all resources you used for this assignment. If
  you worked with people, list them here as well. It is not enough to
  say that you used a resource for help, you need to be specific on the
  link and \emph{how} it was helpful. W/R/T gen AI tools (including GPT,
  etc. ) you cannot use them to do work on your behalf -- you cannot put
  in any of the questions, etc. You can ask for help on logic / sample
  problems. If you do use GPT or other AI tools, you need to provide a
  link to your chat transcript. Any suspected academic integrity
  violations will be immediately reported.
\end{itemize}

\end{document}
