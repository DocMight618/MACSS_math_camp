% Options for packages loaded elsewhere
\PassOptionsToPackage{unicode}{hyperref}
\PassOptionsToPackage{hyphens}{url}
\documentclass[
]{article}
\usepackage{xcolor}
\usepackage[margin=1in]{geometry}
\usepackage{amsmath,amssymb}
\setcounter{secnumdepth}{5}
\usepackage{iftex}
\ifPDFTeX
  \usepackage[T1]{fontenc}
  \usepackage[utf8]{inputenc}
  \usepackage{textcomp} % provide euro and other symbols
\else % if luatex or xetex
  \usepackage{unicode-math} % this also loads fontspec
  \defaultfontfeatures{Scale=MatchLowercase}
  \defaultfontfeatures[\rmfamily]{Ligatures=TeX,Scale=1}
\fi
\usepackage{lmodern}
\ifPDFTeX\else
  % xetex/luatex font selection
\fi
% Use upquote if available, for straight quotes in verbatim environments
\IfFileExists{upquote.sty}{\usepackage{upquote}}{}
\IfFileExists{microtype.sty}{% use microtype if available
  \usepackage[]{microtype}
  \UseMicrotypeSet[protrusion]{basicmath} % disable protrusion for tt fonts
}{}
\makeatletter
\@ifundefined{KOMAClassName}{% if non-KOMA class
  \IfFileExists{parskip.sty}{%
    \usepackage{parskip}
  }{% else
    \setlength{\parindent}{0pt}
    \setlength{\parskip}{6pt plus 2pt minus 1pt}}
}{% if KOMA class
  \KOMAoptions{parskip=half}}
\makeatother
\usepackage{graphicx}
\makeatletter
\newsavebox\pandoc@box
\newcommand*\pandocbounded[1]{% scales image to fit in text height/width
  \sbox\pandoc@box{#1}%
  \Gscale@div\@tempa{\textheight}{\dimexpr\ht\pandoc@box+\dp\pandoc@box\relax}%
  \Gscale@div\@tempb{\linewidth}{\wd\pandoc@box}%
  \ifdim\@tempb\p@<\@tempa\p@\let\@tempa\@tempb\fi% select the smaller of both
  \ifdim\@tempa\p@<\p@\scalebox{\@tempa}{\usebox\pandoc@box}%
  \else\usebox{\pandoc@box}%
  \fi%
}
% Set default figure placement to htbp
\def\fps@figure{htbp}
\makeatother
\setlength{\emergencystretch}{3em} % prevent overfull lines
\providecommand{\tightlist}{%
  \setlength{\itemsep}{0pt}\setlength{\parskip}{0pt}}
\usepackage{bookmark}
\IfFileExists{xurl.sty}{\usepackage{xurl}}{} % add URL line breaks if available
\urlstyle{same}
\hypersetup{
  pdftitle={PSET 6: Discrete random variables},
  hidelinks,
  pdfcreator={LaTeX via pandoc}}

\title{PSET 6: Discrete random variables}
\author{}
\date{\vspace{-2.5em}}

\begin{document}
\maketitle

\textbf{SURVEY!!} Please fill this out:
\url{https://forms.gle/Kn4iPTpUbfqymErG6}

\textbf{Note: all homework uploads should be as a PDF \emph{and} have
the questions identified}. We'll be giving zero credit for submissions
that don't follow this protocol as it adds considerable time to grading.
Thank you!

\begin{itemize}
\item
  Name
\item
  How long did this problem set take you?
\item
  How difficult was this problem set? very easy 1 2 3 4 5 very
  challenging
\end{itemize}

\section{\texorpdfstring{Calculate probabilities in a sample space
\(S\)}{Calculate probabilities in a sample space S}}\label{calculate-probabilities-in-a-sample-space-s}

Events \(A\) and \(B\) are contained within a sample space \(S\). Given
that \(\Pr(A)=0.65\), \(\Pr(B)=0.3\) and \(\Pr(A \cap B) = 0.1\),
find:\footnote{Inspired by Grimmer HW 8.4}

\begin{enumerate}
\def\labelenumi{\alph{enumi}.}
\item
  \(\Pr(A \cup B)\)

  \strut \\
  \strut ~\\
  \strut ~\\
  \strut ~\\
  \strut \\
\item
  \(\Pr(A \cap B^c)\)

  \strut \\
  \strut ~\\
  \strut ~\\
  \strut ~\\
  \strut \\
\item
  \(\Pr[(A \cap B^c) \cup (B \cap A^c)]\)

  \strut \\
  \strut ~\\
  \strut ~\\
  \strut ~\\
  \strut \\
\end{enumerate}

\section{Random variables}\label{random-variables}

We have been discussing random variables. Provide the following
explanations in your own words:

\begin{enumerate}
\def\labelenumi{\alph{enumi}.}
\tightlist
\item
  What is a random variable?
\item
  What is the difference between upper case and lower case x? How (if at
  all) do they matter?
\item
  If I'm thinking about something like \(p_X(x_0)\), what do all the
  parts/pieces mean?
\end{enumerate}

\section{Survey Says}\label{survey-says}

A survey has 54\% respondents 50 or older and 46\% respondents under 50.
Within the survey, on a particular question, 9.5\% of the 50-plus
population agrees strongly while 2.7\% of under 50 respondents agree
strongly.

\begin{enumerate}
\def\labelenumi{\arabic{enumi}.}
\item
  What is the probability someone selected at random is 50 or older? ~
  ~\\
  \strut ~\\
  \strut ~\\
  \strut \\
\item
  The selected individual strongly agrees with the survey question. Now
  what is the likelihood that person is 50 or older? Explain your
  reasoning and SHOW ALL YOUR WORK ~ ~\\
  \strut ~\\
  \strut ~\\
\item
  Are the two answers above the same or different? Explain. ~ ~\\
  \strut ~\\
  \strut ~\\
\item
  (for fun, no points) What is the survey question?
\end{enumerate}

\section{PMF vs CMF}\label{pmf-vs-cmf}

Consider the following function: \(f(x) = \frac{1}{6}\). Find the pmf
and cmf of the function and provide them in a table below.

\hfill\break
\hfill\break
\hfill\break
\hfill\break
\hfill\break
\hfill\break
\hfill\break

\section{Getting a traffic ticket}\label{getting-a-traffic-ticket}

You drive to work 5 days a week for a full year (50 weeks), and with
probability \(p=0.04\) you get a traffic ticket on any given day,
independent of other days. Let \(X\) be the total number of tickets you
get in the year.\footnote{Inspired by BT 2.41}

\begin{enumerate}
\def\labelenumi{\alph{enumi}.}
\item
  What is the probability that the number of tickets you get is exactly
  equal to the expected value of \(X\)?
\end{enumerate}

\hfill\break
\hfill\break
\hfill\break
\hfill\break
\hfill\break
\hfill\break
\hfill\break

\begin{enumerate}
\def\labelenumi{\alph{enumi}.}
\item
  Calculate approximately the probability in (a) using a Poisson
  approximation.
\end{enumerate}

\hfill\break
\hfill\break
\hfill\break
\hfill\break
\hfill\break
\hfill\break
\hfill\break

\hfill\break
\hfill\break
\hfill\break
\hfill\break
\hfill\break
\hfill\break
\hfill\break

\section{Obtaining requests for
information}\label{obtaining-requests-for-information}

\(X\) is a discrete random variable. It takes the value of the number of
days required for a governmental agency to respond to a request for
information. \(X\) is distributed according to the following
PMF:\footnote{Inspired by Grimmer HW10.2}

\[f(x) = e^{-6} \dfrac{6^x}{x!} \mbox{ for } X \in \{0,1,2...\}\]

\begin{enumerate}
\def\labelenumi{\alph{enumi}.}
\item
  Given this information, what is the probability of a response from the
  agency in 5 days or less?

  \hfill\break
  \hfill\break
  \hfill\break
  \hfill\break
  \hfill\break
  \hfill\break
  \hfill\break
\item
  What is the probability the agency response takes more than 10 but
  less than 13 days?

  \hfill\break
  \hfill\break
  \hfill\break
  \hfill\break
  \hfill\break
  \hfill\break
  \hfill\break
\item
  What is the probability the agency response takes more than 5 days?

  \hfill\break
  \hfill\break
  \hfill\break
  \hfill\break
  \hfill\break
  \hfill\break
  \hfill\break
\item
  Suppose using \(X\) you generate a new variable, \textbf{Responsive}.
  \textbf{Responsive} equals 1 if an agency responds in 5 days or less
  and 0 otherwise. What is the expected value of \textbf{Responsive}?

  \hfill\break
  \hfill\break
  \hfill\break
  \hfill\break
  \hfill\break
  \hfill\break
  \hfill\break
\item
  What is the variance of \textbf{Responsive}?

  \hfill\break
  \hfill\break
  \hfill\break
  \hfill\break
  \hfill\break
  \hfill\break
  \hfill\break
\end{enumerate}

\section{Modeling electoral outcomes}\label{modeling-electoral-outcomes}

Suppose we've developed a model predicting the outcome of the upcoming
midterm elections in a state with 4 Congressional districts. In each
district there are two candidates, a Republican and a Democrat. We have
reason to believe the following PMF describes the distribution of
potential election results where \(K \in \{0,1,2,3,4\}\) and is the
number of seats won by Republican candidates in the upcoming election.

\[
\Pr(K=k | \theta) = \binom{4}{k} \theta^{k} (1-\theta)^{4-k}
\]

Based on polling information, we think the appropriate value for
\(\theta\) is 0.423.\footnote{Inspired by Grimmer HW10.3. Data from
  \[generic polling\](\url{https://www.realclearpolling.com/polls/state-of-the-union/generic-congressional-vote})}

\begin{enumerate}
\def\labelenumi{\alph{enumi}.}
\item
  What's the expected number of seats Republicans will win in the
  upcoming election?
\end{enumerate}

\hfill\break
\hfill\break
\hfill\break
\hfill\break
\hfill\break
\hfill\break
\hfill\break
\hfill\break
\hfill\break
\hfill\break

\begin{enumerate}
\def\labelenumi{\alph{enumi}.}
\item
  Given this PMF, what's the probability that no Republican legislators
  win in the upcoming election?

  \hfill\break
  \hfill\break
  \hfill\break
  \hfill\break
  \hfill\break
  \hfill\break
  \hfill\break
  \hfill\break
  \hfill\break
  \hfill\break
\item
  What's the probability that Republican legislators win a majority of
  the seats in this state?

  \hfill\break
  \hfill\break
  \hfill\break
  \hfill\break
  \hfill\break
  \hfill\break
  \hfill\break
\item
  A prominent political pundit declares they are certain that
  Republicans will win a majority of seats in the next election and
  offers the following bet. If Republicans win a majority of the seats,
  we must pay the pundit \$15.00. If Republican's fail to win a majority
  of states, we will win \$20.00. Based on our model, should we take
  this bet? \textbf{Hint: Think of the betting outcomes as a random
  variable. Find the expected value of this random variable.}

  \hfill\break
  \hfill\break
  \hfill\break
  \hfill\break
  \hfill\break
  \hfill\break
  \hfill\break
  \hfill\break
  \hfill\break
  \hfill\break
\item
  Suppose we are offered a second bet with a more complicated structure.
  In this case we'll receive \$100 if the Republicans win a majority,
  \$50 if neither party wins a majority and we'll have to pay \$200 if
  the Democrats win a majority. Should we take this bet?
\end{enumerate}

\section{AI and Resources statement}\label{ai-and-resources-statement}

Please list (in detail) all resources you used for this assignment. If
you worked with people, list them here as well. It is not enough to say
that you used a resource for help, you need to be specific on the link
and \emph{how} it was helpful. W/R/T gen AI tools (including GPT, etc. )
you cannot use them to do work on your behalf -- you cannot put in any
of the questions, etc. You can ask for help on logic / sample problems.
If you do use GPT or other AI tools, you need to provide a link to your
chat transcript. Any suspected academic integrity violations will be
immediately reported.

\end{document}
