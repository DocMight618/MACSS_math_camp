% Options for packages loaded elsewhere
\PassOptionsToPackage{unicode}{hyperref}
\PassOptionsToPackage{hyphens}{url}
\documentclass[
]{article}
\usepackage{xcolor}
\usepackage[margin=1in]{geometry}
\usepackage{amsmath,amssymb}
\setcounter{secnumdepth}{5}
\usepackage{iftex}
\ifPDFTeX
  \usepackage[T1]{fontenc}
  \usepackage[utf8]{inputenc}
  \usepackage{textcomp} % provide euro and other symbols
\else % if luatex or xetex
  \usepackage{unicode-math} % this also loads fontspec
  \defaultfontfeatures{Scale=MatchLowercase}
  \defaultfontfeatures[\rmfamily]{Ligatures=TeX,Scale=1}
\fi
\usepackage{lmodern}
\ifPDFTeX\else
  % xetex/luatex font selection
\fi
% Use upquote if available, for straight quotes in verbatim environments
\IfFileExists{upquote.sty}{\usepackage{upquote}}{}
\IfFileExists{microtype.sty}{% use microtype if available
  \usepackage[]{microtype}
  \UseMicrotypeSet[protrusion]{basicmath} % disable protrusion for tt fonts
}{}
\makeatletter
\@ifundefined{KOMAClassName}{% if non-KOMA class
  \IfFileExists{parskip.sty}{%
    \usepackage{parskip}
  }{% else
    \setlength{\parindent}{0pt}
    \setlength{\parskip}{6pt plus 2pt minus 1pt}}
}{% if KOMA class
  \KOMAoptions{parskip=half}}
\makeatother
\usepackage{graphicx}
\makeatletter
\newsavebox\pandoc@box
\newcommand*\pandocbounded[1]{% scales image to fit in text height/width
  \sbox\pandoc@box{#1}%
  \Gscale@div\@tempa{\textheight}{\dimexpr\ht\pandoc@box+\dp\pandoc@box\relax}%
  \Gscale@div\@tempb{\linewidth}{\wd\pandoc@box}%
  \ifdim\@tempb\p@<\@tempa\p@\let\@tempa\@tempb\fi% select the smaller of both
  \ifdim\@tempa\p@<\p@\scalebox{\@tempa}{\usebox\pandoc@box}%
  \else\usebox{\pandoc@box}%
  \fi%
}
% Set default figure placement to htbp
\def\fps@figure{htbp}
\makeatother
\setlength{\emergencystretch}{3em} % prevent overfull lines
\providecommand{\tightlist}{%
  \setlength{\itemsep}{0pt}\setlength{\parskip}{0pt}}
\usepackage{bookmark}
\IfFileExists{xurl.sty}{\usepackage{xurl}}{} % add URL line breaks if available
\urlstyle{same}
\hypersetup{
  pdftitle={PSET 7: General random variables},
  hidelinks,
  pdfcreator={LaTeX via pandoc}}

\title{PSET 7: General random variables}
\author{}
\date{\vspace{-2.5em}}

\begin{document}
\maketitle

\textbf{Note: all homework uploads should be as a PDF \emph{and} have
the questions identified}. We'll be giving zero credit for submissions
that don't follow this protocol as it adds considerable time to grading.
Thank you!

\begin{itemize}
\item
  Name
\item
  How long did this problem set take you?
\item
  How difficult was this problem set? very easy 1 2 3 4 5 very
  challenging
\end{itemize}

\section{Identifying the PDF}\label{identifying-the-pdf}

A recent college graduate is moving to Des Moines, Iowa to take a new
job, and is looking to purchase a home. When searching for properties on
the real estate websites, it is possible to select the price range of
housing in which one is most interested. Suppose the potential buyer
specifies a price range of \$150,000 to \$300,000, and the result of the
search returns a thousand homes with prices distributed uniformly
throughout that range. Identify \(E[X]\) and \(\sigma\) of the
probability density function associated with this random
variable.\footnote{Inspired by Stinerock 6.1}

\section{Moments}\label{moments}

We discussed moments in class and the slides. Suppose someone were
asking you about our class content. How would you explain the following
in a non-technical way? Approx 2 sentences each.

\begin{enumerate}
\def\labelenumi{\alph{enumi}.}
\tightlist
\item
  What is an expected value and how does it relate to a moment?
\item
  What is the first moment and why do we care about it?
\item
  What is the second moment and how does it add additional information
  to the first moment?
\end{enumerate}

\section{Calculating ideal points}\label{calculating-ideal-points}

Suppose An is a voter living in the country of Freedonia, and suppose
that in Freedonia, all sets of public policies can be thought of as
representing points on a single axis (e.g.~a line running from more
liberal to more conservative). An has a certain set of public policies
that they want to see enacted. This is represented by point \(v\), which
we will call An's \textbf{ideal point}. The utility, or happiness, that
An receives from a set of policies at point \(l\) is \(U(l)=-(l-v)^2\).
In other words, An is happiest if the policies enacted are the ones at
their ideal point, and they get less and less happy as policies get
farther away from this ideal point. When they vote, An will pick the
candidate whose policies will make them happiest. However, An does not
know exactly what policies each candidate will enact if elected -- they
have some guesses, but can't be certain. Each candidate's future
policies can therefore be represented by a continuous random variable
\(L\) with expected value \(\mu_l\) and variance \(Var(L)\).\footnote{Inspired
  by Grimmer HW11.2}

\begin{enumerate}
\def\labelenumi{\alph{enumi}.}
\item
  Express \(E(U(L))\) as a function of \(\mu_l\), \(Var(L)\), and \(v\).
  Why might we say that An is \textbf{risk averse} -- that is, that An
  gets less happy as outcomes get more uncertain?

  \hfill\break
  \hfill\break
  \hfill\break
  \hfill\break
  \hfill\break
  \hfill\break
\item
  Suppose An is deciding whether to vote for one of two candidates:
  Dwight Schrute or Leslie Knope Suppose An's ideal point is at 1,
  Schrute's policies can be represented by a continuous random variable
  \(L_S\) with expected value at 1 and variance equal to 6, and Knope's
  policies can be represented by a continuous random variable \(L_K\)
  with expected value at 3 and variance equal to 1. Which candidate
  would An vote for and why? What (perhaps surprising) effect of risk
  aversion on voting behavior does this example demonstrate?
\end{enumerate}

\section{Parliamentary elections}\label{parliamentary-elections}

After an election in a parliamentary system, a government (consisting of
a prime minister and a cabinet) is formed by gathering the support of a
majority of newly elected members of parliament. Typically a government
is allowed to remain in power for a certain number of years before new
elections must be called. However, elections can be held earlier if the
Parliament passes a vote of no confidence or the prime minster decides
to dissolve the government. Suppose we are studying Country Z (which
uses a parliamentary system) and we are interested in the duration of
governments. In Country Z, governments must call elections at least
every 5 years, but they could be called sooner if there is a vote of no
confidence or the prime minister dissolves the government. Let the
continuous random variable \(X\) denote the amount of time (measured in
years) between the last election and the calling of the next election.
\(X\) has support on all real numbers between 0 and 5. Suppose we know
that \(X\) has the probability density function

\[
f(x) = \begin{cases}
kx^4 &  0 < x < 4 \\
0 & \text{otherwise}
\end{cases}
\]

where \(k\) is some constant.

\begin{enumerate}
\def\labelenumi{\alph{enumi}.}
\item
  Find \(k\).
\item
  Find the CDF of \(X\).
\item
  Find \(E(X)\) and \(Var(X)\).
\item
  Find the median of \(X\) (the value of \(x\) at which
  \(\Pr(X \leq x) = \frac{1}{2}\)).
\item
  What is the probability that the government remains in power for
  exactly 5 years? Why?
\item
  What is the probability that the government remains in power between 3
  and 5 years?
\item
  What is the probability that the government remains in power for less
  than 3 years or more than 5 years?
\end{enumerate}

\section{Calculating the CDF}\label{calculating-the-cdf}

Z is distributed according to the following PDF

\[
f(z) = \begin{cases}
\gamma \exp(-\gamma z) &  0 \le z \\
0 & \text{otherwise}
\end{cases}
\]

\begin{enumerate}
\def\labelenumi{\alph{enumi}.}
\item
  What is \(F(z)\), the CDF of this distribution?
\item
  Using your answer to the previous question, evaluate the CDF for the
  interval from 7 to 12.
\item
  Suppose \(\gamma\) is 3. Given this, what is \(q\), the 10th
  percentile value of Z?
\item
  We observe a single random draw from \(Z\), what is the probability
  this observation is less than .5? Again suppose that \(\gamma\) = 3.
\end{enumerate}

\section{Working with normal random
variables}\label{working-with-normal-random-variables}

Let \(X\) and \(Y\) be normal random variables with means 0 and 1,
respectively, and variances 1 and 4, respectively.\footnote{Inspired by
  BT 3.11}

\begin{enumerate}
\def\labelenumi{\alph{enumi}.}
\item
  Find \(\Pr(X \leq 1.5)\) and \(\Pr(X \leq -1)\).
\item
  Find the pdf of \((Y - 1) / 2\).
\end{enumerate}

\section{Spot the CDF/pdf}\label{spot-the-cdfpdf}

From the following, identify functions as either pdfs, CDFs, or neither.

\begin{enumerate}
\def\labelenumi{\alph{enumi}.}
\item
  \[
  f(x) = \begin{cases}
  x^2 &  0 < x < 4 \\
  0 & \text{otherwise}
  \end{cases}
  \]
\item
  \[
  f(x) = \begin{cases}
  0.25 &  0 < x < 4 \\
  0 & \text{otherwise}
  \end{cases}
  \]
\item
  \[
  f(x) = \begin{cases}
  0.25x &  0 < x < 4 \\
  0 & \text{otherwise}
  \end{cases}
  \]
\item
  \[
  f(x) = \begin{cases}
  x^2 &  0 < x < 1 \\
  0 & \text{otherwise}
  \end{cases}
  \]
\end{enumerate}

\section{Functions}\label{functions}

We discussed the following functions in class: Uniform Exponential,
Gamma, Normal, \(\chi^2\), Student's t

Provide a one-sentence summary of the general shape and expected value
of:

\begin{enumerate}
\def\labelenumi{\alph{enumi}.}
\tightlist
\item
  Exponential
\item
  Normal
\item
  Student's t
\end{enumerate}

\section{AI and Resources statement}\label{ai-and-resources-statement}

Please list (in detail) all resources you used for this assignment. If
you worked with people, list them here as well. It is not enough to say
that you used a resource for help, you need to be specific on the link
and \emph{how} it was helpful. W/R/T gen AI tools (including GPT, etc. )
you cannot use them to do work on your behalf -- you cannot put in any
of the questions, etc. You can ask for help on logic / sample problems.
If you do use GPT or other AI tools, you need to provide a link to your
chat transcript. Any suspected academic integrity violations will be
immediately reported.

\end{document}
