\documentclass[11pt]{article}

\usepackage{amsmath}
\usepackage{amsfonts}
\usepackage{amssymb}
\usepackage{amsthm}
\usepackage{graphicx}
\usepackage{graphicx}
\usepackage{tikz}
\usepackage{pgfplots}
\usepackage{setspace}
\usepackage{fullpage}
\usepackage{color}
\usepackage{xcolor,colortbl}
\usepackage{comment}
\usepackage{bm}
\usepackage{url}
\usepackage{hyperref}
\usepackage{color,hyperref}
\definecolor{darkblue}{rgb}{0.0,0.0,0.3}
\hypersetup{colorlinks,breaklinks,
            linkcolor=darkblue,urlcolor=darkblue,
            anchorcolor=darkblue,citecolor=darkblue}

%Change to Helvetica
%\usepackage[scaled]{helvet}
%\renewcommand*\familydefault{\sfdefault} %% Only if the base font of the document is to be sans serif
%\usepackage[T1]{fontenc}

%Change to fourier
%\usepackage{fourier}
%\usepackage[T1]{fontenc} 

\usepackage[T1]{fontenc}
\usepackage{charter}
\usepackage[expert]{mathdesign}

\begin{document}
\noindent \textbf{Problem Set 2 \color{gray} ANSWER KEY} \color{black}

\begin{enumerate}
\item Summation
\begin{enumerate}
\item $\sum_{n=1}^{7}3$ \color{gray} $7*3=21$\color{black}
\item $\sum_{n=0}^{4} 2n+8$ \color{gray} $\sum_0^4 2n+ \sum_0^4 8= \frac{2*4(5)}{2}+5*8=60$ Note that you need to include when $n=0$  \color{black}
\end{enumerate}

\item Limits and Continuity
\begin{enumerate}
\item What is a limit? \color{gray} A limit is when we focus on what will ultimately happen as a variable (usually x) approaches some pre-determined value. \color{black}
\item Why do we care? \color{gray} We care for three reasons: 1) it helps us determine if functions are continuous, 2) we can use limits to understand where a function is headed, even if it's not defined at a certain point, 3) we'll use them for derivatives in particular as we ask about the rate of change at a particular slice of the function, as the slice gets smaller and \small smaller and \tiny smaller. \normalsize \color{black}
\item What is a continuous function? \color{gray} Informally, it is a function that can be drawn without picking a pencil up from the paper. More formally, it is where the function is defined at every point in the domain and where, for any point in that domain, where the limit from the left equals the limit from the right and these equal the value at that point. \color{black}
\end{enumerate}

\item Set theory
\begin{enumerate}
\item In roster notation, write the set characterized in set-builder notation as $S = \{x \in \mathbb{Z}, 2 <x <5\}$. \color{gray} $\{ 3,4\}$\color{black}
\item Graph on the number line the interval $[-3, 2)$. \color{gray} Need open circle\color{black}
\item Is the following statement True or False? $\forall x \in S, x\geq 2$ for $S = \{3,2,5,9\}$. \color{gray} T. All values are greater than or equal to 2. \color{black}
\item Is the following statement True or False? $\exists x \in S\; s.t. \; x \notin \mathbb{Z}$ for $S = \{3,2,5,9\}$. \color{gray} F. All numbers in the set are integers and thus no element is not a member of the set of integers. \color{black}
\item Is $\{1,2,3,4\}$ a subset of $\{4,3,1,2\}$? Is it a proper subset? \color{gray} It is a subset, but not proper subset because there are no elements in the second that are not elements of the first set. \color{black}
\item Using logical symbols (including $\exists$ and $\forall$) write the definition of a proper subset.  \color{gray} Took most answers as long as they looked `mathy'. Wanted some version of $A \subset B \leftrightarrow  x \in A \rightarrow x\in B \wedge \exists y\in B | y \notin A $ \color{black}
\item If $A = \{\text{soup}, 8\}$ and $B = \{x, \text{soup}\}$ find $A \cup B$. \color{gray} $A \cup B= \{ \text{soup}, 8, x \}$\color{black}
\item (Follow-up): Now find $A \cap B$. \color{gray} $A \cap B= \{ \text soup\}$\color{black}
\item (Follow-up): Find the Cartesian Product $A \times B$. \color{gray} $A \times B = \{ (\text{soup}, x), (\text{soup}, \text{soup)}, (8, x), (8, \text{soup})\}$ \color{black}
\end{enumerate}

\item Review:
\begin{enumerate}
\item Write out `6 choose 3' mathematically and solve. \color{gray} $\binom{6}{3}=\frac{6!}{3!3!}=\frac{6*5*4}{3*2*1}=20$ \color{black}

\item Add these two matrices: $
\begin{bmatrix}
   2 & 4 & 2 \\
    1 & 4 & 0 \\
    2 & 6 & 0
\end{bmatrix}
+
\begin{bmatrix}
5 & 1 & 1 \\
2 & 2 & 2\\
4 & 1 & 3
\end{bmatrix}
$
{\color{gray}=$\begin{bmatrix}
7 & 5 & 3 \\
3 & 6 & 2\\
6 & 7 & 3
\end{bmatrix}$}


\item Multiply these two matrices: $
\begin{bmatrix}
   2 & 4 & 2 \\
    1 & 4 & 0 \\
    2 & 6 & 0
\end{bmatrix}
*
\begin{bmatrix}
5 & 1 & 1 \\
2 & 2 & 2 \\
4 & 1 & 3
\end{bmatrix}
$
{\color{gray} $=\begin{bmatrix}
26 & 12 & 16 \\
13 & 9 & 9\\
22 & 14 & 14
\end{bmatrix}
$}

\item  List three things you got wrong on yesterday's assignment. What are they, what is the correct response and how will you address/fix this moving forward? \color{gray} Responses will vary. \color{black}
\end{enumerate}

\begin{comment}
\item Introduction to differentiation
\begin{enumerate}
\item What is a derivative? Why might we find it useful? \color{gray} A derivative enables us to do many different things, primarily understand rates of change at particular points and patterns of change overall. It also enables us to find maxima and minima. \color{black}

\item Give an example of a derivative we might care about (think about the education and salary graphs from lecture) \color{gray} Many examples work: Arrival/exit in the workforce and national GDP, disease rates, vaccination rates, just about anything that you can look at over time. \color{black}

\item If $f(x) = x^2$ and $g(x)= 3x + 9$, write $z(x) = f(g(x))$ and simplify (i.e. expand). \color{gray} $z(x)=(3x+9)^2=9x^2+54x+81$ \color{black}

\item Compute manually, using the formula for a derivative (i.e. $ lim_{h\rightarrow0}\frac{f(x+h)-f(x)}{h}$), the derivative of $x^3$. \color{gray} $lim_{h\rightarrow0}\frac{(x+h)^3-x^3}{h}=lim_{h\rightarrow0}\frac{x^3+3x^2h+3xh^2+h^3-x^3}{h}=lim_{h\rightarrow0}\frac{3x^2h+3xh^2+h^3}{h}=lim_{h\rightarrow0}3x^2+3xh+h^2=3x^2$ \color{black}


\item Find the derivative of $f(x) = 2x^2 + 7x + 9$. \color{gray} $f'(x)=4x+7$ \color{black}

\item Find the derivative of $f(x) = 3x^2$ \color{gray} $f'(x)=6x$ \color{black}
\end{enumerate}



\end{comment}
\end{enumerate}

\end{document}




